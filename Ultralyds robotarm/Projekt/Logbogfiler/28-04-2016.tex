\chapter{Logbog}
\section{Dato: 28-04-2016}
\hrule
\textbf{Fremmødte Nina, Ida og Anne. Freja, Mette og Ditte (Viborg)} \\
\textbf{Fraværende: } \\
\textbf{Ansvarlig:} Nina og Ida  \\
\textbf{Dagens dagsorden: }
\begin{enumerate}
	\item Ida, Nina og Anne skal til møde med Søren på Navitas
	\item Ida Nina og Anne skal til vejledermøde med Lene (se også referat for samme sag)
	\item Ditte, Mette og Freja tager til Viborg og foretager interview
	\item Evt. 
\end{enumerate}

\underline{\textbf{Logbog}}\\
Møde med Søren Pallesen
\begin{itemize}
\item Vil gerne hænge det op i et stativ for at få større trykkraft. Ikke i loftet, da den ikke er fleksibel så. 
\item De er i tvivl om, hvilken løsning de vælger – arm på stativ eller deltarobot. 
Prototype i stativ med en UR3. 
\end{itemize}
\begin{itemize}
\item Lavet målinger i forhold til, hvor hårdt sonografer trykker. Ved en bagoverbøjet livmoder trykkes der 10-11 kg. Alle andre scanninger var 2 kgs tryk. 
\item Arbejdsstillingen gør, at de føler, de trykker med stor kraft. 
\end{itemize}
\begin{itemize}
\item Nedbringe mængden af de 2 kilos tryk – og så de tunge scanninger skulle de selv foretage manuelt. Eventuelt dække 80 procent af alle scanninger med robotten. 
\item Overvægtige: Tryk, hudfolder og BMI
\item Kan de levere et produkt, der kan klare det? Uklart 
\item Hvis den bliver hængt op, kan den trykke med større kraft. 
\end{itemize}
\begin{itemize}
\item Ved bagoverbøjet livmoder skal man scanne gennem tarmene. Anatomiske opbygninger kan også gøre, at det er hårdere – så ikke kun overvægt.
\end{itemize}
\begin{itemize}
\item Tidligst i 2017, at de har et produkt. 
\item Kamera i produktet – til den telemedicinske løsning. 
\item Pris: cirka 400.000 kroner. Produktionsprisen vil være billigere ved en deltarobot. 
\item Skejby sygehus: 15 sonografer. 
\item Uddannelse til at benytte joysticket: 
\item Certificering for brug af systemet. 
\item Sikkerhedsstop – afbryder selv ved maksimum tryk. 
\item Sygedage: Skejby
\item Cirka fire sygedage på et år. 
\item Afskriver produktet: 10 år. 
\end{itemize}
Møde med Lene
\begin{itemize}
\item Se referat
\end{itemize}
Interview i Viborg
\begin{itemize}
\item Se interview ark i bilag
\item Fik et langt interview med Karen Marie og Tove, og fik set scanninger
\end{itemize}
\newpage