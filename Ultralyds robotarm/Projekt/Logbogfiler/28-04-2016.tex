\chapter{Logbog}
\section{Dato: 28-04-2016}
\hrule
\textbf{Fremmødte Nina, Ida og Anne. Freja, Mette og Ditte (Viborg)} \\
\textbf{Fraværende: } \\
\textbf{Ansvarlig:} Nina og Ida  \\
\textbf{Dagens dagsorden: }
\begin{enumerate}
	\item Ida, Nina og Anne skal til møde med Søren på Navitas
	\item Ida Nina og Anne skal til vejledermøde med Lene (se også referat for samme sag)
	\item Ditte, Mette og Freja tager til Viborg og foretager interview
	\item Evt. 
\end{enumerate}

\underline{\textbf{Logbog}}\\
Møde med Søren Pallesen\\
Status på produktet:
\begin{itemize}
\item Vil gerne hænge UR3 op i et stativ for at få større trykkraft. Ikke i loftet, da den ikke er fleksibel så - systemet skal gerne være mobilt. 
\item De er i tvivl om, hvilken løsning der bliver valgt i sidste ende, da der er mange ændringer fra uge til uge – arm på stativ eller deltarobot. \\
Vi vælger at gå med prototype i stativ med en UR3 robotarm. 
\end{itemize}
Undersøgelse af tryk:
\begin{itemize}
\item Søren har været ude og foretage målinger i forhold til, hvor hårdt sonografer trykker. Har er et dynamometer blevet tilsluttet proben, hvormed det kan måles hvor mange kilo sonografen trykker med. Ved en bagoverbøjet livmoder trykkes der 10-11 kg. Alle andre scanninger var 2-3 kgs tryk. 
\item Arbejdsstillingen gør, at de føler, de trykker med stor kraft. 
\item Undersøgelsen blev foretaget på 5 sonografer på én arbejdsdag.
\end{itemize}
Systemes funktionalitet
\begin{itemize}
\item Nedbringe mængden af de 2 kilos tryk – og så de tunge scanninger skal stadig fortages manuelt af sonograferne. Systemet vil derfor kun komme til at dække 70-80 procent af alle scanninger på de gravide. 
\item Nogle af de der ikke vil kunne blive scannet og derfor ligger i de 20-30 procent: Overvægtige: Tryk, hudfolder og BMI. Det er nemlig uklart om systemet vil kunne komme til at klare scanningerne på denne gruppe.
\item Hvis UR3 robotarmen bliver hængt op, kan den trykke med større kraft. 
\item Ved bagoverbøjet livmoder skal man scanne gennem tarmene. Anatomiske opbygninger kan også gøre, at det er hårdere – så ikke kun overvægt. Derfor ligge denne gruppe af kvinder også i de 20-30 procent.
\item Sikkerhedsstop – afbryder selv ved maksimum tryk. 
\item Produktet afskrives efter 10 år
\begin{itemize}
\item Der sættes en grænse, sådan at systemet/robotarmen selv slår fra når sonografen påvirker joysticket med et vis tryk.
\item Dette er en sikkerhedsgrænse.
\item Grænsen kan sættes efter behov.
\end{itemize}
\item Sonograferne skal uddannes til at benytte joysticket: Certificering for brug af systemet.
\end{itemize}
Hvor ender produktet:
\begin{itemize}
\item Produktet vil tidligst være klar i 2017 - på grund af udvikling, godkendelse mm. 
\item Kamera i produktet – kun til den telemedicinske løsning.  
\item Prisen for endelig produkt ligger endnu på cirka 400.000 kroner. Produktionsprisen vil være billigere ved en deltarobot. 
\end{itemize}
Skejby:
\begin{itemize}
\item Søren har selv været ude og snakke med Skejby sygehus: 15 sonografer. 
\item Undersøgt antal af sygedage i Skejby: Cirka fire sygedage på et år.  
\end{itemize}
Møde med Lene
\begin{itemize}
\item Se referat
\end{itemize}
Interview i Viborg
\begin{itemize}
\item Se interview ark i bilag
\item Fik et langt interview med Karen Marie og Tove, og fik set scanninger
\end{itemize}
\newpage