\chapter{Mødereferat}

\section{Dato: 19-04-2016}
\hrule
\textbf{Fremmødte: } Freja, Mette, Nina, Ida, Anne og Ditte. Lene \\
\textbf{Fraværende:} \\
\textbf{Referent: } Nina \\
\textbf{Dagens dagsorden: }
\begin{enumerate}
	\item Godkendelse af referat fra sidst
\item Status på MTV-rapport, hvad er blevet skrevet?
\item Status på kontakt til Skejby afdeling
\item Udbytte fra besøg i Horsens
\begin{itemize}
\item Blandt andet: spørgeskemaer droppet pga. for få sonografer
\end{itemize}
\item  Hvordan refereres der til kilder?
\begin{itemize}
\item Specielt personer og uofficielle interviews/samtaler med sonografer.
\end{itemize}
\item Teknologi afsnit: Hvad er fremgangsmåden når produktet ikke er færdig lavet?
\begin{itemize}
\item Må det bygges på antagelser og forventninger fra Søren Pallesen?
\end{itemize}
\item Hvordan udarbejdes en sammenkobling mellem interview og artikler?
\item Revurdere tidsplan
\item Evt.
\end{enumerate}

\textbf{Opgaver:} \newline
Til næste gang:
\begin{itemize}
\item De der ikke var med i Horsens læser dokumenter igennem.
\end{itemize}
\textbf{Referat:}
\begin{itemize}
\item Referat fra sidst er godkendt – Ida har rettet i "fremmødte". 
\item Etikdelen er blevet godkendt af Preben. 
\item Lave antagelser i teknologiafsnittet $\rightarrow$ skal fremgå meget tydeligt, at det er antagelser. 
\item Skal have fastlagt møde med Søren i næste uge.
\item Forskellige holdninger fra Horsens $\rightarrow$ holde artikler og holdninger op mod hinanden. 
\begin{itemize}
\item Sammenlign artikler med Horsens
\item Være kritisk i forhold til de enkeltes situation
\end{itemize}
\item Organisationsafsnit: fakta afsnit om, hvordan de enkelte scanninger foregår (Horsens) $\rightarrow$ sammenholde udtalelser/konklusioner fra artikler og Horsens $\rightarrow$ evt. en tabel
\item Nyt aspekt: Stigende overvægt. 
\item Danmarks statistik omkring gravide og overvægtige $\rightarrow$ se hvordan udviklingen er. Tage med i baggrundsafsnittet/perspektiveringen. 
\item Mails skal lægges som bilag. Projektadministration $\rightarrow$ hører til processen. 
\item Kontakt til Samuel igen vedrørende Skejby
\item Glade for jobbet som sonograf $\rightarrow$ tager arbejdsgener som en del af jobbet . Stil spørgsmålstegn ved denne påstand.
\item Kilder: Referat fra mødet skal lægges som interview bilag. Lydoptagelsen skal gemmes, men ikke lægges med. 
\item Ingen citat når vi bruger det fra Horsens $\rightarrow$ skrive hvor stor afdelingen er og hvem der har udtalt sig. Ingen henvisning til kilde $\rightarrow$ man skriver "på baggrund af dette afsnit osv." 
\item Priser fra Carsten Riis $\rightarrow$ skriv om til et estimat/anslået pris. Refererer til ham.
\item Gøre det klart i rapporten, at vi ikke har kunne finde artikler til patient og økonomi afsnittet. 
\item Reviderer selv tidsplan.
\item Argumentere for, hvad der har mest fokus for vores arbejde:
\begin{itemize}
\item Organisation
\item Patient
\item Teknologi
\item Økonomi
\end{itemize}
\end{itemize}

\newpage


