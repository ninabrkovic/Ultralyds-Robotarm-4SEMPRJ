\section{Dato: 10-03-2016}
\hrule
\textbf{Fremmødte: } Mette, Ida, Nina (Kom senere), Anne, Freja og Ditte. Samuel og Lene \\
\textbf{Fraværende: } \\
\textbf{Referent: } Ida \\
\textbf{Dagens dagsorden: }
\begin{enumerate}
	\item Gennemgang af spørgeskema
	\item Gennemgang af interviewspørgsmål
	\item Gennemgang af mail
	\item Evt. 
\end{enumerate}

\textbf{Opgaver:}
\begin{itemize}
\item Til næste gang: 
\begin{itemize}
\item Kontaktinformationer sendes til Samuel 
\item Referater sendes til Lene. 
\item Vi skal have Sørens specialeopgave. 
\item Udarbejde søgeprotokol
\item Udarbejde tidsplan
\end{itemize} 
\end{itemize} 
\textbf{Referat:}\\
\begin{itemize}
\item Vi har fået nogle input til vores spørgeskema af Bente (igår). 
\item Samuel skaber først en kontakt til Anette Jensen på Skejby inden, for at hun ved hvem vi er.
\item Mere imødekommende i Horsens, der har tidligere været aftale med at Horsens skulle teste udstyret.
\item Vi skal i kontakt med både Horsens (Tina) og Skejby (Anette)
\begin{itemize}
\item Godt at have to afdelinger for dokumentation.
\end{itemize}
\end{itemize}
\textbf{Spørgeskema:}
\begin{itemize}
\item Der skal være en indledning, der fortæller hvad vores formål er med at udsende spørgeskemaet.
\item Stil spørgsmål til deres stilling, samt hvor længe de har haft denne funktionalitet.
\item Vent med at skrive intervaller på til vi har snakket med Tina og Anette.
\item Positivt at vi er påbegyndt med skemaet, vi har fået skabt et godt udgangspunkt, der kan arbejdes videre med.
\item Vend rækkefølgen på svarmuligheder om, så det positive kommer først.
\end{itemize}
\textbf{Mail og interview:}
\begin{itemize}
\item Samuel har sendt dokumenterne tilbage med kommentarer.
\item Vær opmærksom på om det vi udsender til interview er vores formuleringer til et interview eller blot er som udgangspunkt for hvad vi ønsker at få besvaret i løbet af et interview. For gruppen er det et udgangspunkt til det som ønskes besvaret til et interview.
\item De kommer ikke til at kunne svare på økonomiske spørgsmål, som hvor meget en ultralydsscanning koster for deres afdeling.
\item Vi kan få svar på tid osv. altså organisations prægede spørgsmål.
\item Vi mangler et spørgsmål om oplæring af sonografer
\item Carsten Riis, Indkøber medicoteknisk afdeling Aarhus Universitetshospital Skejby, ved muligvis godt hvor længe en ultralyd scanner holder og hvad de koster.
\item Omformuleringer af spørgsmål for at klargøre, at vi ved noget - altså ikke kommer uforberedte.
\item Første møde skal være indledende og for at skabe en god relation. (Danne et godt indtryk)
\item Vi skal gerne have lov til at lave et observationsstudie: observere en scanning, samt dagsprocedurer. Samtidig kan sådan en dag også bruges til at validere de tal som en oversygeplejerske muligvis er kommet med.
\begin{itemize}
\item Her skal vi have klargjort inden, hvad vi skal have ud af denne observation
\item Her kan andre spørgsmål besvares - interview med den enkelte sonograf.
\item Se en normal scanningsdag og hvilke typer scanninger der foretages.
\end{itemize}
\end{itemize}
\textbf{Søgning:}
\begin{itemize}
\item I vores artikelsøgning skal vi søge mere generelt. Da der ikke er lavet artikler på ultralyds scanning med robotarm på gravide endnu.
\item Referencer: Tal i firkantet parentes. Vancouver.
\item Vi vil i starten få mere end 100 hits. Vi skal gå ind og beskrive hvordan vi har fået søgningen specificeret og dermed begrænset.
\end{itemize}
\textbf{Evt.}
\begin{itemize}
\item Vi skal have lavet en tidsplan
\end{itemize}
\newpage