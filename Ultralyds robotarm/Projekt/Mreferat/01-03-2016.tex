\section{Dato: 01-03-2016}
\hrule
\textbf{Fremmødte: Freja, Mette, Ida, Nina, Anne og Ditte. Lene Haüser og Søren Pallesen} \\
\textbf{Fraværende:)} \\
\textbf{Referent: Nina } \\
\textbf{Dagens dagsorden: }
\begin{enumerate}
	\item Brugen og funktionen af ultralyds robotarmen, v. Søren
	\item Problemstillinger og udfordringer ift. indførelse af teknologien på nuværende tidspunkt 
	\item Vende projekts fokus – Hvilke inputs har du til vinklen?
	\item Fremvise udarbejdet interessentanalyse – Har du tilføjelser/ændringer hertil?
	\item Kontakt til ”Afdeling Kvindesygdomme og fødsler” på AUH
	\item Evt. 
\end{enumerate}

\textbf{Opgaver:} \newline

\textbf{Referat:}\\
\textbf{Robotic Ultrasound}\\
Ultralydsscanninger over afstand – assistentstyret system og et robotstyret. \\\\
To produkter som udgangspunkt:
\begin{itemize}
\item Robotic Ultrasound Ergo
	\begin{itemize}
	\item Ultralydsscanninger af afhængige af den operatør, der udfører det og den som tolker det
	\item Personalet er belastet ved at lave disse scanninger – arbejdsskader
	\item Tiltag i dag: arbejder på nedsat tid – Horsens sygehus arbejder de 20 timer med patienter
	\item Tager lang tid at uddanne personalet
	\item Mange har fået forbud om at foretage scanninger pga. Arbejdsskader
	\item Restriktioner
	\end{itemize}
\item Telemedicinsk produkt: 
	\begin{itemize}
	\item Samme joystick og robot – lægen skal være uafhængig af andet personale mht. knapper osv. 
	\item Imødekomme udfordringer i relation til det nye super sygehus, hvor man centraliserer specialer. 
	\end{itemize}
\end{itemize}
\begin{itemize}
\item Løsning: Robotarm koblet til en ultralydsscanner - deres egen. Joystick. Der bliver påmonteret en fiktiv ultralydsprobe på, den skal monteres ned oven på. Knapperne kommer til at skulle bevæge x-y-z og anden knap skal kunne inkludere rotationsbevægelser. Probeholderen skal være en universel probeholder – alle kan gå i. Dynamometer – vil bygge deres egen (måle tryk 10-15 kg). 
\item Hvor hårdt trykker man? 
\item Fravalgt Universal Robots.
\item Software, som styrer systemet $\Rightarrow$ udviklet af bachelor studerende i efteråret. 
\item Forhandlinger med et firma om en ny ultralydsscanner robotarm. Lige nu er der ingen prototype – han låner udstyr fra sygehuse på nuværende tidspunkt.
\end{itemize}
\textbf{Arbejdsskader:}
\begin{itemize}
\item Ingen videnskabelige artikler. 
\item Litteratur fra MTC (gammelt).
\item Undersøgelser og små projekter fra sygehusafdelinger.
\item Medarbejdere har ikke øget sygefravær. Dedikeret i deres job. 
\end{itemize}
\begin{itemize}
\item Kan det betale sig økonomisk? 
\item Salgspris på 400.000 kroner til den ergonomiske del. 
\item Opvejning $\Rightarrow$ 400.000 kroner kan man få en medarbejder for. 
\item Hvornår er produktet tjent ind igen?
\item Telemedicinske del $\Rightarrow$ første salg bliver til forskningsprojekt. Grønland 2017.
\item Ergo løsning $\Rightarrow$ 2016. Tage en eksisterende robot og tilpasse deres produkt til den.
\end{itemize}
\textbf{Godkendelse af robotarm til medicinsk brug:}
\begin{itemize}
\item Svære del. Her skal der bruges flest penge – den største udfordring.
\item Hvis produktet ikke kan medicinsk godkendes er der ingen salg. 
\item Spørg efter Samuels slides.
\item Når hele systemet ikke er medicinsk godkendt, må det ikke testes. 
\item Vil også gerne have godkendelse til deres amerikanske marked.
\end{itemize}
\begin{itemize}
\item 4 elementer:
\begin{itemize}
\item Teknologi
\item Økonomi 
\item Organisation (arbejdsskader, besparelser, interviews på sygehuset med personale)
\begin{itemize}
\item Kvindesygdomme og Fødsler (Skejby) – find afdelingssygeplejerske
\item Horsens sygehus – afdelingssygeplejerske
\end{itemize}
\item Borger
\end{itemize}
\end{itemize}
\textbf{Scanninger:}
\begin{itemize}
\item Sonografer – jordemødre og sygeplejersker, der har taget en efteruddannelse. De sidder med mange scanninger. 
\item Hvordan bliver man sonograf? Ressourcer.
\end{itemize}
\textbf{Patienter:}
\begin{itemize}
\item Spørge ind til, hvad patienter vil tænke omkring robotarm scanning. Patientoplevelse.
\item Tage højde for elektromagnetisk stråling. Virksomhed i Odense. 
\item Hvordan er strukturen og hierarkiet? Lettere i Horsens end i Skejby.
\end{itemize}
\textbf{Ultralydsscanninger ud over gravide:}
\begin{itemize}
\item Mavebrok, hjertescanning (15 procent), radiologien og abdominal (20 procent), muskler, skelet (sener, led, knæ). Scanning tager typisk 20 minutter. Man skal holde styr på en ledning. Dårlig arbejdsstilling. 
\item 1,7 mio. ultralydsscanninger om året i Danmark. Den billigste, let tilgængelig. 
\item 25 procent er gravide scanninger.
\end{itemize}

\newpage