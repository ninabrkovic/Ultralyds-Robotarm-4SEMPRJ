\chapter{Mødereferat}

\section{Dato: 12-05-2016}
\hrule
\textbf{Fremmødte: Ida, Nina, Ditte, Mette, Freja og Anne. Lene og Samuel} \\
\textbf{Fraværende:} \\
\textbf{Referent: Nina } \\
\textbf{Dagens dagsorden: }
\begin{enumerate}
\item Gennemgang af rettelser
\item Status, hvordan ser det ud fra vejleders synspunkt? Er vi langt nok, er der noget vi mangler at tage højde for og lignende.
\item Evt.
\end{enumerate}
\textbf{Spørgsmål til møde}
\begin{itemize}
\item Hvordan laver vi reference til Søren og interview?
\item Indledning: Billede af opstilling?
\item Overblik over økonomi og de to scenarier
\end{itemize}
\textbf{Forside}
\begin{itemize}
\item Navn på MTV'en og billede på en forside. 
\item Navn og studienummer på en anden side – titelblad. 
\end{itemize}
\textbf{Referencer: }
\begin{itemize}
\item Udtalelser fra Søren: må gerne skrive citater fra ham. Referer til et referat fra møde med Søren. Skrive ind, at Søren har været ude og måle det efter ude på Skejby med 5 personer. 
\item Interviews: referer til bilag hvor interviews er i. Skriv det efter første statement eller til sidst i afsnittet. Ved videnskabelige artikler bør det skrives første gang, det bliver brugt.  
\item Referer til, hvor oplysningen står i. Kilde – Søren Pallesen. 
\end{itemize}
\textbf{Indledning:}
\begin{itemize}
\item Have billede i indledning af opstillingen, så censor kan se, hvad det drejer sig om. Enten billede eller opstilling. Eller billede af før og efter side om side. Gamle billede stillet op mod det nye. Det er fint, hvis vi har billede på både forside og i indledning. 
\end{itemize}
\textbf{Økonomi:}
\begin{itemize}
\item To scenarier – nutidige og fremtidige. Overblik – billede, figur eller tabel. Hvad består ændringen i. Figuren/tabellen skal være i starten af afsnittet. 
\item Snak med 3D gruppen $\Rightarrow$ gode flowcharts. Farver: IKKE rød og grøn (haha). 
\item Skriv om, at vi har forsøgt at finde information omkring arbejdssygedage grundet skader. Hvad koster en sygedage generelt. Skriv, at vi ikke har fundet data på det, men at vi har forsøgt. Find tal på, at de stopper på job før tid.
\end{itemize}
\textbf{Konklusion:}
\begin{itemize}
\item Konkludere på resultaterne. Fordele og ulemper op mod hinanden. 
\end{itemize}
\textbf{Diskussion og perspektivering: (diskussion)}
\begin{itemize}
\item Snakke telemedicinsk. Fremtidsmæssigt. 
\item Stigende BMI: hvor meget kommer robotarmen til at kunne hjælpe, hvis BMI'en stiger. "Hvem skal løfte folderne?" Komplicerede scanninger. Reference – dokumentation. Hvad ville have overbevist os? Hvad mangler vi for at kunne sætte to røde streger under resultatet. 
\item Ting, som vi ikke får afdækket i rapporten, er til diskussionsafsnit. Begrænsninger af vores studie og hvad vi kunne have tænkt os. Perspektivering plejer at være en del af diskussionen. 
\end{itemize}
\textbf{Status:}
\begin{itemize}
\item Referencer skal der være styr på
\item Skal have noget ind med videnskabelige artikler (fra Søren)
\item BMI skal ind 
\item I perspektivering kan der komme referencer ind med telemedicin 
\item Artikel om hjerte ultralydsscanning med joystick = kan bruges under diskussionen eller baggrund. Kan det understøtte Sørens udtalelser? Tiden om scanningen. Indbygget bias i undersøgelsen. 
\item Nævne de andre artikler med den store ramme under teknologiafsnittet. 
\end{itemize}
\newpage


