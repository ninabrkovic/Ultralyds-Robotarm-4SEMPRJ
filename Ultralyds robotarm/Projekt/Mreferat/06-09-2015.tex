\chapter{Mødereferat}

\section{Dato: 18-02-2016}
\hrule
\textbf{Fremmødte: Freja, Mette, Ida, Anne og Ditte. Begge vejledere} \\
\textbf{Fraværende: Nina(kom senere)} \\
\textbf{Referent: Ditte } \\
\textbf{Dagens dagsorden: }
\begin{enumerate}
	\item Have intro til projektet
	\item Forventningsafstemme mellem vejleder og grupper
	\item Aftale fast vejledermødedag fra efter påsken
	\item Evt. 
\end{enumerate}

\textbf{Opgaver:} \newline

\textbf{Referat:}
\begin{itemize}
	\item Samuel snakker om selve projektet. Jordmødre skal ændre på deres arbejdsstillinger. 
	Det er testet sammen med Grønland og gynops afdeling (vej Skejby ca.). Hvis der findes oplysninger om enten tele eller arbejdsstillinger skal de bare medtages. 
	\item Søren er uddannet Fysioterapeut. Hvad vil han gerne have. Vær objektiv.  Alle mails til ham skal gå gennem vejleder(Lene). 
	\item Søgning af tekster, vi får undervisning om hvordan og hvad der skal skrives med ved søgning. Bare prøv i starten og ret det til. 
	Forklar hvordan vi sorterer i artikler, finder evt 100 og bruger kun 5. Hvorfor? metode beskrivelse. Forklar valg(inklusiv) og fravalg(eksklusiv). 
	\item Arbejde nu: finde frem til et overordnet problemstilling (telemedicin eller arbejdsstillinger). Evt. tage kontakt til Grønland for spørgsmål. 
	\begin{itemize}
		\item hvad betyder det for de fire emner
		\item hvad siger patienten og lignende 
		\item De afhænger af emnet
	\end{itemize}   
	\item Den bliver brugt til (grønland): gravide, blindtarme(ondt i marven eller blindtarms betændelse) - rent økonomisk 
	\begin{itemize}
		\item til differential diagnose eller centralisering af centrene og derude kan fx sosu'er skanne ude i udkanten.
	\end{itemize} 
	\item Vejleder Rolle: Lene er primær, men Samuel vil gerne hjælpe og vil meget gerne være indover. 
	\item Tænk over hvordan vi udveksler oplysninger,  alle skal kunne vide hvad der står. Alle skal være med i to emner for at undgå det. Dele artikler med resten af gruppen. 
	\item Interviews er en god øvelse og en god idé, hvis der ikke er nok oplysninger om emnet. 
	\begin{itemize}
		\item Jordmødre har ikke prøvet ar arbejde med det, men det skal de lærer. 
		\item Hvad tænker de om det. 
	\end{itemize}
	\item Deadline i maj: Det vi nu har nået. Ikke et færdigt produkt, så der er tid til rettelser.
	\item Aftale møde med Søren, for at snakke emnet igennem. Lene skal spørges om hun vil med. Samuel vil gerne. 
	\item Aftale møde med jordmødre: vejledere skriver først. De vil gerne med (Når det bliver aktuelt). 
	\item Lav fælles mail konto - faglig. 
	\item har været nede og se joystick og robotarm i Shannon.  
\end{itemize}

\newpage


