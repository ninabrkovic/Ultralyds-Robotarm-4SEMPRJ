\chapter{Mødereferat}

\section{Dato: 18-02-2016}
\hrule
\textbf{Fremmødte: Freja, Mette, Ida, Anne og Ditte. Begge vejledere} \\
\textbf{Fraværende: Nina(kom senere)} \\
\textbf{Referent: Ditte } \\
\textbf{Dagens dagsorden: }
\begin{enumerate}
	\item Have intro til projektet
	\item Forventningsafstemme mellem vejleder og grupper
	\item Aftale fast vejledermødedag fra efter påsken
	\item Evt. 
\end{enumerate}

\textbf{Opgaver:} \newline

\textbf{Referat:}
\begin{itemize}
	\item \textbf{Projektet formål:} Samuel snakker om selve projektet og dets formål. Jordemødre skal ændre på deres arbejdsstillinger da nuværende metoder giver arbejdsskader. Det er her ultralyds robotarmen kommer i spil.
	Det er testet sammen med Grønland og gynops afdeling (ved Skejby Sygehus). Hvis der findes oplysninger om enten tele eller arbejdsstillinger skal de bare medtages. Vi vælger selv vinkel på opgaven, men det er mest oplagt at vælge jordemoder og arbejdsskader.
	\item Søren er uddannet Fysioterapeut. Hvad vil han gerne have. Vær objektiv over for ham og produktet.  Alle mails til ham skal gå gennem vejleder(Lene). 
	\item \textbf{Litteratursøgning:} Søgning af tekster, vi får undervisning om hvordan og hvad der skal skrives med ved søgning. Bare prøv i starten og ret det til. 
	Forklar hvordan vi sorterer i artikler, finder evt 100 og bruger kun 5. Hvorfor? metode beskrivelse. Forklar valg(inklusiv) og fravalg(eksklusiv). Arbejdet her vil primært lægge efter påske når den nødvendige undervisning er modtaget.
	\item \textbf{Videre arbejde:} Finde frem til et overordnet problemstilling (telemedicin eller arbejdsstillinger). Evt. tage kontakt til Grønland for spørgsmål, hvis denne vinkel vælges. 
	\begin{itemize}
		\item Hvad betyder det for de fire emner
		\item Hvad siger patienten og lignende 
		\item Spørgsmålene afhænger af vinklen på projektet
	\end{itemize}   
	\item Den bliver brugt til (grønland): gravide, blindtarme(ondt i maven eller blindtarmsbetændelse) - rent økonomisk 
	\begin{itemize}
		\item Til differential diagnose eller centralisering af centrene og så kan fx sosu'er eller sygeplejersker skanne ude i udkantsbygder.
	\end{itemize} 
	\item \textbf{Vejlederrolle:} Lene er primær, men Samuel vil gerne hjælpe og vil meget gerne være indover. 
	\item Tænk over hvordan vi udveksler oplysninger,  alle skal kunne vide hvad der står. Alle skal være med i to emner for at undgå det. Dele artikler med resten af gruppen. 
	\item Interviews er en god øvelse og en god idé, hvis der ikke er nok oplysninger om emnet. Dette vil i højst sandsynlig komme til at skulle udføre.
	\begin{itemize}
		\item Jordemødre har ikke prøvet ar arbejde med det, men det skal de lærer. 
		\item Hvad tænker de om det mv. er mulige spørgsmål der kan arbejdes videre med.
	\end{itemize}
	\item \textbf{Deadlines:} Eneste givet deadline fra vejleder er 4. maj - Aflever det vi nu har nået på daværende tidspunkt. Ikke et færdigt produkt, så der er tid til rettelser.
	\item Aftale møde med Søren, for at snakke emnet igennem. Lene skal spørges om hun vil med. Samuel vil gerne, hvis tid.
	\item Aftale møde med jordemødre: Vejledere skriver først med udgangspunkt med udkast fra os. De vil gerne med (Når det bliver aktuelt). 
	\item Lav fælles mail konto - Bliver oprettet. Al kommunikation vil fremover foregå gennem denne.
	\item Har været nede og se joystick og robotarm i Shannon, RobotLab.  
\end{itemize}

\newpage


