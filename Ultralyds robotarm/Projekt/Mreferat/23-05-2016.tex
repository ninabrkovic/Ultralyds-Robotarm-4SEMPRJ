\chapter{Mødereferat}

\section{Dato: 23-05-2016}
\hrule
\textbf{Fremmødte: Ida, Nina, Mette, Freja og Anne. Lene} \\
\textbf{Fraværende:} \\
\textbf{Referent: Nina } \\
\textbf{Dagens dagsorden: }
\begin{enumerate}
\item MTV-spørgsmål: skal det med, hvad er det, hvor skal det være?
\item Interessantanalyse; skal den med?
\item Metode afsnit, hvor langt skal det være, hvor dybtegående skal det være?
\item Evt.
\end{enumerate}

\begin{itemize}
\item MTV-spørgsmål: skal det med, hvad er det, hvor skal det være?
\begin{itemize}
\item Skal fremgå i indledning/baggrund. "Belyse dette emne ud fra disse spørgsmål".
\item Spørgsmål er en hjælp til læseren til, hvad de skal læse om nu. 
\end{itemize}
\item Interessantanalyse; skal den med?
\begin{itemize}
\item Skal med som bilag. Nævn den i projektafgrænsning og referere til bilaget. 
\end{itemize}
\item Metode afsnit, hvor langt skal det være, hvor dybdegående skal det være?
\begin{itemize}
\item Figur om inklusions- og eksklusionskriterier. Metodeafsnittet kunne også være i starten af de fire hovedområder. Argumentere for, hvorfor vi har valgt, som vi har gjort. 
\item Hvordan kunne vi have fået relevant viden ind, hvis vi kunne finde det? Hvad har vi søgt efter af videnskabelige artikler – mangler proces i metodeafsnittet. 
\item Afsnit om spørgeskema skal flyttes ned under Organisationsafsnit. 
\item Interviews skal stå generelt i starten af afsnittet – semistruktureret + begrundelse hertil. 
\item I hvert enkelt afsnit kan interviews uddybes
\end{itemize}
\end{itemize}
\begin{itemize}
\item Problemer med referencer (URL).
\item Figur i indledning af selve systemet:
\begin{itemize}
\item Må gerne være samme figur som er længere nede i rapporten. Dette er en hjælp til læseren. 
\end{itemize}
\item Resumé:
\begin{itemize}
\item Hvad står der i opgaven? Resumerer vores projekt – de vigtigste ting, vi har fundet. 
\begin{itemize}
\item Indledning
\item Metode
\item Konklusion 
\item Perspektivering/diskussion
\end{itemize}
\end{itemize}
\end{itemize}

\newpage


