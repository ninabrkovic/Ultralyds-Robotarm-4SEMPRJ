\chapter{Abstract} 
\subsubsection{Introduction}
Sonographers who work with ultrasonic examination of pregnant women are at risk of getting work-related disorder due to poor work postures. It is often necessary to press the ultrasonic probe against the stomach, while the probe is being held at arm’s length. An Ultrasonic Robotic Arm will reduce the amount of poor work postures. The Robotic Arm where the ultrasonic probe is attached will be controlled by the sonographer with a joystick. The sonographer thereby avoids the previously mentioned physical challenges and potential discomforts. 

\subsubsection{Methods}
The purpose of this report has been to study which consequences the implementation of the Ultrasonic Robotic Arm can have with focus on four perspectives: Technology, Organization, Patient, and Economics. 
Information which can be used in all perspectives has been obtained through interviews with departments for ultrasonic examinations of pregnant women in the Regional Hospital of Horsens and the Regional Hospital of Viborg. Specific methods has been used for each perspective.

\subsubsection{Discussion}
This report is based on a summary of interviews, scientific articles, and assumptions. This is partly because the Ultrasonic Robotic Arm is not fully developed, which causes an insecurity which could have been reduced throughout other researches. 
The Ultrasonic Robotic Arm can be used as equipment for telemedicine in the future. This will require an attachment of a camera and a microphone to the system. There will be no direct changes to the Ultrasonic Robotic Arm. The sonographer does not need to be in the same room as the pregnant woman during the ultrasonic examination. 

\subsubsection{Conclusion} 
The Ultrasonic Robotic Arm will be able to reduce the amount of work-related disorder and poor work postures for sonographers during ultrasonic examination of pregnant women. The sonographers will be able to perform ultrasonic examinations 37 hours a week and stay in the job for a greater amount of years. The Robotic Arm will not lead to changes for the patient with regard to the quality and the result of the ultrasonic examination. 
The Ultrasonic Robotic Arm has a technological restriction, which means it can be used for 70-80\% of the ultrasonic examinations of pregnant women. The remaining 20-30\% of the ultrasonic examinations must be performed manually. From an economic perspective savings in wage costs will not cover the annual depreciation provision on the purchase of the Ultrasonic Robotic Arm.


