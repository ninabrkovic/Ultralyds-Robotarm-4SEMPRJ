\chapter{Abstract} 
\subsubsection{Introduction}
Sonographers who work with ultrasound examination of pregnant women are at risk of getting work-related discomfort due to awkward positioning. It is often necessary to press the ultrasound probe against the stomach, while the probe is being held at arm’s length. An Ultrasonic Robotic Arm will reduce the amount of awkward work postures. The Robotic Arm, where the ultrasonic probe is attached, will be controlled by the sonographer by a joystick. The sonographer thereby avoids the previously mentioned physical challenges and potential discomforts. 

\subsubsection{Methods}
The purpose of this report has been to study which consequences the implementation of the Ultrasonic Robotic Arm can have with focus on four elements: Technology, Organization, Patient and Economics. 
Information which can be used in all perspectives has been obtained through interviews with departments for scans of pregnant women in The Hospital Unit Horsens and Regional Hospital Midt Viborg. There has been used specific methods for each perspective.

\subsubsection{Discussion}
This report is based on a summary of interviews, scientific articles and assumptions. This is partly because the Ultrasonic Robotic Arm is not fully developed, which causes an insecurity that could have been reduced through other researches. 
The Ultrasonic Robotic Arm can be used as telemedicine equipment in the future. This will require a connection of a camera and a microphone with no direct changes of the system. The sonographer does not need to be in the same room as the the pregnant woman. 

\subsubsection{Conclusion} 
The Ultrasonic Robotic Arm will be able to reduce the amount of work-related discomforts and awkward work postures for sonographers during ultrasound examination of pregnant women. The sonographers will be able to perform scans 37 hours a week and work longer. The Robotic Arm will not lead to changes for the patient with regard to the quality and the result of the examination. 
The Ultrasonic Robotic Arm has a technological restriction, which means it can be used for 70-80\% of the scans of pregnant women. The remaining 20-30\% of the scans must be performed manually. Savings in labor costs will from an economic perspective not cover the annual depreciation provision on the purchase of the Ultrasonic Robotic Arm.


