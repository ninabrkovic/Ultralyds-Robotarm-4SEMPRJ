\chapter{Perspektivering}

I 2007 blev en fremtidig plan fremlagt for landets sygehuse. Her var målet at skære ned på antallet af sygehuse, der har akutmodtagelse døgnet rundt. Antallet skulle gå fra 40 i 2007 til 21 i 2020. Denne strukturændring skal give færre og mere specialiserede sygehuse, hvilket skal gøre at behandlingen bedre og mere effektiv for de danske borgere. Dog betyder dette at der bliver mere afstand mellem sygehusene for borgerne, da de mindre sygehuse i provinserne lukker. Dette giver mere transport for borgerene ved rutinemæssige undersøgelser, herunder bl.a. ved ultralydsscanning af gravide kvinder. Her vil Ultralyds Robotarmen fremtidsmæssigt have et stort potentiale som en telemedicinsk løsning, hvor problemstillingen med afstande fjernes. Denne løsning vil derfor også kunne afhjælpe situationen i bl.a. Grønland, hvor der er endnu større afstande mellem borgere og sygehuse. \cite{greenland}

Den telemedicinske løsning vil kunne imødekomme økonomiske, organisatoriske, samfundsmæssige og patientrelaterede udfordringer. Her vil Ultralyds Robotarmen blive udstyret med kameraer og en mikrofon, som skal transmittere billede og lyd til sonografens placering, se Bilag 12, 28.04.2016. Sonografen vil her via dummy-proben kunne scanne den gravide fra afstand. Sonografen behøver derfor ikke at være placeret i samme rum. Ved den gravide skal der være en assistent, som skal sætte systemet til og forberede scanningen.

I Grønland er der store afstande mellem patient og sonograf, og derfor meget transport mellem steder, hvor der kan udføres ultralydsscanninger. Her vil den telemedicinske løsning kunne bevirke at en sonograf i Danmark, eller et andet sted på Grønland, vil kunne foretage scanningen på afsides steder i Grønland. Dette vil give besparinger på transportudgifter, da patienterne såvel som sonograferne, ikke længere skal transporteres over store afstande ved de rutinemæssige scanninger. 
Dette vil ligeledes være til gavn i Danmark, hvor ultralydsscanninger vil kunne blive udført i provinserne, selvom sonograferne er placeret i de større byer. Derfor er der stort potentiale både internationalt, national og regionalt, da sonografen fra sin egen afdeling vil kunne tilbyde sin ekspertise på tværs af afdelingerne på landets hospitaler. 

Det at Ultralyds Robotarmen vil kunne blive en telemedicinsk løsning underbygges af de studier, som har undersøgt dette aspekt. Undersøgelser inde for telemedicinsk ultralydsscanning har været i gang i en årrække og undersøger, hvorvidt det er muligt at få samme kvalitet på scanningerne som ved manuelle scanninger. Det undersøges om billedkvaliteten forbliver god, når disse skal sendes over nettet, og om selve scanningen kan foretages tilfredsstillende\citep{5}\citep{8}\citep{18}. 





