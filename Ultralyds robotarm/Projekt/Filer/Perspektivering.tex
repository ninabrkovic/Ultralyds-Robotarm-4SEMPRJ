\chapter{Diskussion og perspektivering}
%Evt. større økonomisk analyse med tal\\

Denne mini-MTV rapport bygger bl.a. på data opsamlet igennem interviews fra to af landets ultralydsscanningsafdelinger, HMH og RMV. Dette gør det svært at generalisere for samtlige sygehuse og hospitaler på landsplan. Interviews med flere ultralydsscanningsafdelinger ville derfor give et mere klart billede af, hvordan situationen er på de forskellige afdelinger. Ved implementering af Ultralyds Robotarmen kan det også være meget nyttigt at kende til de enkelte afdelingers strukturer. 
Idet svarene fra de to afdelinger er relativ ens, vil man med en god tilnærmelse kunne drage nogle paralleller til den overordnede opbygning og struktur på ultralydsafdelingerne i Danmark.

En større undersøgelse, som skal vise antal og baggrunden for sonografernes sygemeldinger, ville være et af de næste skridt, der skulle foretages. Det har ikke været muligt at få data på sygefravær i forbindelse med arbejdsrelaterede gener fra de to afdelinger. En sådan undersøgelse kunne derfor være med til at underbygge behovet for implementering af robotarmen.
Efter samtale med sonograferne på afdelingen er holdningen dog, at arbejdet er hårdt fysisk belastende.  De vil derfor ikke kunne holde til arbejdet som sonograf indtil pensionistalderen, med mindre der sker nogle ændringer. Selvom de er meget glade for jobbet, vil de kunne blive nødsaget til at skifte karriere eller stilling. 

% Sonograferne aflastes af Ultralyds Robotarmen, men der vil endnu være behov for ergonomiske værktøjer og redskaber. Scanningerne foretages med samme fremgangsmåde som før, dog er arbejdet fjernet fra strakt arm over den gravide, til arbejde med dummy-proben. Dette gør at sonografen ikke har samme akavede arbejdsstillinger som tidligere. 

Et teststudie vil være en metode, hvormed det kan testes om denne løsning vil kunne fungere på landets afdelinger. Her skal systemet implementeres på en afdeling,hvor brugen følges tæt. Dette studie vil derfor kunne benyttes til at verificere Ultralyds Robotarmens effekt. Sådan et studie har der dog ikke været mulighed for at udføre, da robotarmen endnu ikke er færdigudviklet. 
Afdelingerne som blev interviewet er begge meget åbne overfor teknologien. HEH har indvilliget i at teste robotarmen, når denne er blevet færdigudviklet, og RMV vil gerne være med i et eventuelt fremtidigt samarbejde. 

Stigningen i kvinders BMI kan have en betydning for robotarmens effekt. (reference) Disse scanninger er mere komplicerede og udgører cirka 20\% af det samlede antal scanninger. De komplicerede scanninger skal udføres manuelt. Idet BMI’en forøges, bliver denne procentdel større, og derfor skal flere og flere scanninger foretages manuelt. 

Fremtidsmæssigt ses et stort potentiale ved at gøre Ultralyds Robotarmen til en telemedicinsk løsning. Dette vil kunne imødekomme økonomiske, organisatoriske, samfundsmæssige og patientrelaterede
udfordringer. Her vil Ultralyds Robotarmen blive udstyret med kameraer og en mikrofon, som skal transmittere billede og lyd til sonografens placering. Sonografen vil her via dummy-proben kunne scanne den gravide fra afstand. Sonografen behøver derfor ikke at være placeret i samme rum. Ved den gravide skal der være en assistent, som skal sætte systemet til og forberede scanningen.
 
Dette vil bl.a. kunne afhjælpe situationen i Grønland. Her er der store afstande og derfor meget transport mellem steder, hvor der kan udføres ultralydsscanninger. Her vil den telemedicinske løsning kunne bevirke at en sonograf i Danmark, eller et andet sted på Grønland, vil kunne foretage scanningen på afsides steder i Grønland. Dette vil give besparinger på transportudgifter, da patienterne såvel som sonograferne, ikke længere skal transporteres over store afstande ved de rutinemæssige scanninger. 
Dette vil ligeledes være til gavn i Danmark, hvor ultralydsscanninger vil kunne blive udført i provinserne, selvom sonograferne er placeret i de større byer.
Derfor er der stort potentiale både internationalt, national og regionalt, da sonografen fra sin egen afdeling vil kunne tilbyde sin ekspertise på tværs af afdelingerne på landets hospitaler. 

Det at Ultralyds Robotarmen vil kunne blive en telemedicinsk løsning underbygges af de studier, som har undersøgt dette aspekt. Undersøgelser inde for telemedicinsk ultralydsscanning har været i gang i en årrække og undersøger, hvorvidt det er muligt at få samme kvalitet på scanningerne som ved manuelle scanninger. Det undersøges om billedkvaliteten forbliver god, når disse skal sendes over nettet(reference), og om selve scanningen kan foretages tilfredsstillende(reference). 

