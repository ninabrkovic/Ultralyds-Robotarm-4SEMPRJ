\chapter{Økonomi} \label{Okonomi}

Formålet med dette afsnit er ud fra et økonomisk aspekt, at vurdere - så vidt muligt sort på hvidt - om en given teknologisk løsning er værd at implementere i praksis. I dette tilfælde, gøres det ved at sammenligne de nuværende udgifter på ”Afdelingen for Kvindesygdomme og fødsler” på Skejby Hospital til ultralyds scannings udstyr af gravide med udgifterne til implementering af ultralyds robotarm.  

Det er valgt at tage udgangspunkt i det udstyr der benyttes ved en nakkefoldsscanning i 11. til 13. uge uden komplikationer. Som det fremgår af teknologi-afsnittet er der overlap mellem det udstyr, der benyttes i dag og det der skal benyttes ved robotarmen. Udstyr-overlappet består af C1-5-RS convex transducer, software til avancerede 3D/4D billeder samt printer med tilbehør. 

Vurderingen tager udgangspunkt i nuværende udstyr og arbejdsgange, hvor der er foretaget gennemsnitlige estimater, som sammenlignes ved brug af robotarm. Her er der ligeledes lavet estimater, da robotarmen ikke er færdig udviklet. Disse estimater er lavet ud fra en vurdering af, hvordan robotarmen vil udspille sig i praksis. 

\section{Den nuværende situation}
Uddannelse pris og tid, vedligehold og servicering af udstyr, indkøb af udstyr, antal af sonografer

\section{Den fremtidige situation}
Uddannelse pris og tid, service og vedligehold, indkøb, antal scanninger pr. robot pr. dag. \\
Alt i dette underafsnit bygger på antagelser gjort af Søren Pallesen, Robotic Ultrasound ApS. 

\section{Delkonklusion}

