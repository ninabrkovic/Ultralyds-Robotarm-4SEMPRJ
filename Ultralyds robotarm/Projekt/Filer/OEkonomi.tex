\chapter{Økonomi} \label{Okonomi}
Formålet med dette afsnit er ud fra et økonomisk aspekt at vurdere om en given teknologisk løsning er værd at implementere i praksis. I dette tilfælde gøres det ved at benytte omkostningsminimeringsanalyse. Da det antages, at den sundhedsmæssige effekt er ens i den nuværende situation og i den fremtidige situation, hvor robotarmen implementeres som en add-on løsning til eksisterende ultralydsudstyr. 

Der opstilles to scenarier. Scenarie 1 er den nuværende situation på en ultralydsscannings stue, mens scenarie 2 er den fremtidige situation på en stue med en robotarm. I analysen ønskes det at klarlægge forskellene i de to scenarier, i forhold til hvilke ressource- og omkostningsforhold der er i det enkelte scenarie. Belægget for dette afsnit er skabt pga. interview med ”Kvindeafdelingen, Svangre- og Ultralydsambulatorium” på Hospitalsenheden Horsens (HEH). Her er deres arbejdsgange og arbejdsforhold blevet klarlagt, se Organisation for yderligere beskrivelse heraf. Dermed er det vigtigt at pointere, at forholdene skal være sammenlignelige med HEH førend, at der kan konkluderes tilsvarende for andre afdelinger. 

Yderligere tager analysen udgangspunkt i, at CEO Søren Pallesen hos Robotic Ultrasound forventer, at salgsprisen på robotarmen med tilbehør bliver 400.000 DKK, når den kommer på markedet. Scenarie 2 vil primært bygge på hypoteser, da robotten ikke er færdigudviklet endnu. 

Grundet at robotarmen er en add-on løsning til det eksisterende ultralydsudstyr, er priser på indkøb og vedligehold af ultralydsudstyret ikke medtaget i denne analyse, da det antages at denne må være ens for begge scenarier, og dermed ikke beskriver en forskel. Alle priser i de følgende beregninger er angivet uden moms. 


\section{Scenarie 1 - Den nuværende situation}
I dette scenarie fokuseres der på omkostnings- og ressourceforbruget for at holde en ultralydsscannings stue i drift fem dage om ugen. I den nuværende situation scanner en sonograf fire dage om ugen, mens sonografen den femte dag varetager en række arbejdsopgaver, som sonografen i princippet er overkvalificeret til at varetage. Dette skyldes, at det ønskes at aflaste sonograferne i deres arbejdsforhold, og dermed mindske mængden af arbejdsgener og potentielle arbejdsskader. \\
Omkostningerne i den nuværende situation kan opdeles i følgende punkter. Hver punkt vil blive uddybet og argumenteret senere i afsnittet. 
\begin{itemize}
\item Lønomkostninger
\item Arbejdsgener og -skader
\item Udgifter til forebyggelse
\item Uddannelse af flere sonografer
\end{itemize}
Lønomkostninger er i dette scenarie givet ved 1,2 sonograf. Den 1,2 sonograf er estimeret på baggrund af, at det er nødvendigt, at have en hel sonograf samt 1/5 af en anden sonografs arbejdstid for at have stuen bemandet fem dage om ugen. Månedslønnen for en sonograf med 2 års erfaring med kvalifikationstillæg er på løntrin 6, hvilket giver 26.967 DKK. På årsbasis giver det en årsløn på 323.604 DKK. Lønudgifter er dermed beregnet til:
\begin{equation}
323604 \text{ DKK}\cdot1.2 = 388325 \text{ DKK}
\end{equation}
Det ses også, at i den nuværende organisering af arbejdsgangen er der taget konsekvensen af, at scanning af gravide er et belastende arbejde. Dette viser sig gennem en række omkostninger til forebyggelse af arbejdsskader og -gener. Det koster blandt andet penge, at en sonograf ikke kan scanne fuldtid, og dermed bliver nødsaget til at påtage sig opgaver, sonografen er overkvalificeret til. Disse opgaver vil være billigere omkostningsmæssigt, hvis de bliver varetaget af en person, hvis kvalifikationsniveau passer til opgaven. Denne person vil typisk modtage en lavere løn end sonografen, og dermed vil det føre til en besparelse. 

Samtidig er der udgifter til forebyggelse såsom ergonomiske stole, elastik-træning, massage i arbejdstiden og wellness-konsulenter, der altid står til rådighed, for at give sonograferne råd og vejledning om bedre arbejdsstillinger og -forhold. 

Yderligere er der omkostninger forbundet med uddannelse af sonografer. Uddannelse af en sonograf er estimeret til at koste 108.000 DKK. Uddannelsen foregår som mesterlære over en 16 ugers periode. I denne periode er den nye sonograf altid under vejledning af mesteren. Dermed er der dobbeltbemanding på hver scanning. Således antages det at prisen på uddannelsen er givet ved lønomkostninger til den ekstra mand i form af mesteren i de 16 uger:
\begin{equation}
27000 \text{ DKK}\cdot4 \text{ måneder} = 108000 \text{ DKK}
\end{equation}
Der er dog også forbundet omkostninger med, at sonografen først antages at kunne foretage alle typer scanninger hundrede procent på egen hånd efter to år. Dermed kan der være forlængede scanningstider for den nye sonograf, såfremt vedkommende støder på ukendte ting og bliver nødsaget til at opsøge hjælp fra mere erfarne sonografer. 

Fra et regionsperspektiv er der ikke direkte omkostninger forbundet med, at en sonograf pådrager sig en arbejdsskade grundet dårlige arbejdsforhold. Hvorimod fra et samfundsperspektiv vil det kunne føre til afledte omkostninger, i form af at personen bliver udkørt af arbejdet, og dermed bliver tvunget tidligt på pension. Dermed vil denne person være mindre værd, grundet at denne persons samlede livsløn vil være lavere end en sonograf der har været på arbejdsmarkedet et fuldt arbejdsliv. Dette fører til at denne person koster samfundet penge, fremfor at bidrage til samfundet. 

Hvor stort et problem arbejdsskader er økonomisk, er svært at måle. En arbejdsskade viser sig som en smerte, men det svært at angive smerteværdien i kroner og øre. Yderligere er det svært at svare på om smerten fremkommer af scannings arbejdet eller af en fritidsinteresse sonografen har. Dette gør at arbejdsskader viser sig som afledte omkostninger.

\section{Scenarie 2 - Den fremtidige situation}
I dette scenarie fokuseres der på de ressourcer der vil komme i spil ved implementering af en Ultralyds Robotarm som en add-on løsning til eksisterende ultralydsudstyr. Der tages ligeledes udgangspunkt i omkostnings- og ressourceforbruget for at holde en ultralydsscannings stue i drift fem dage om ugen. I dette afsnit vil der blive trukket paralleller til scenarie 1, for at tydelige gøre hvor omkostnings forskellene er. \\
Omkostningerne i den fremtidige situation er givet ved følgende punkter. Hvert punkt vil blive uddybet i afsnittet.
\begin{itemize}
\item 400.000 DKK til robotarm med tilbehør og stativ
\item Færre arbejdsskader og -gener
\item Ingen udgifter til forebyggelse
\item Regionens ansvar for personale og arbejdsmiljø
\item Lønomkostninger
\end{itemize}
Etableringsomkostninger til robotarmen med tilbehør er estimeret til at være på 400.000 DKK. For de fleste institutioner vil en udgift på 400.000 DKK være et stort udlæg, derfor er det mere relevant at fordele omkostninger over den årrække, som indkøberne afskriver teknologien over. Afskrivnings perioden er på ti år, da det er estimeret at udstyret er forældet efter ti år. Eksisterende ultralydsudstyr afskrives ligeledes over ti år. 

Fordeles etableringsomkostningerne over ti år efter annuitetsmetoden med forrentningsfaktor på 2,2 \%, se Forkortelser og formler, formel 1. Forrentningsfaktoren er estimeres til at være et gennemsnit af inflations renten i Danmark i 2016 og 2020 \cite{inflation}. Årligt giver dette en omkostning på 18.355 DKK:
\begin{equation}
\left(\frac{(1+0.022)^{10}\cdot0.022}{(1+0.022)^{10}-1}\right)\cdot400000 \text{ DKK}=18355 \text{ DKK}
\end{equation}

Det forventes at ved brug af en robotarm ved scanning vil belastningen på sonografen være markant sænket. Det skyldes at sonografen ikke bliver belastet af at påføre store tryk på patienten, samt bevæge arm og skulder ud i dårlige arbejdsstillinger. Dette uddybes i afsnittet Aktører under Organisation \ref{aktoerer_organisation}. Dermed antages i denne analyse, at der i dette scenarie ikke vil opstå arbejdsgener eller -skader grundet scanningsarbejdet. 

Det bevirker at udgifterne til forebyggelse af sonograferne forsvinder. Afdelingen vil dermed ikke have udgifter til ressourcer, som velfærdskonsulenter, ergonomiske stole, aflastende arbejdstider, massage og elastik-træning. I dette punkt adskiller scenarie 1 sig meget fra scenarie 2. Ved besparelse på forebyggelse kan det frigive flere penge til sundhedsfremmende løsninger.

I det første scenarie medførte det i princippet ingen omkostninger for regionen, hvis personalet bliver arbejdsskadet og dermed sygemeldt. Hvorimod det vil være dyrt for samfundet, grundet udbetaling af understøttelse og manglende skatte indkomst. I scenarie 2 er det lige omvendt. Regionen vil have udgiften til robotarmen på 400.000 DKK som en meromkostning, hvilket vil være et stort udlæg hvis der udelukkende ses på tallene. Samtidig har regionen også et ansvar for dens personale og arbejdsmiljø, herunder sikkerhed og sundhed. Dette ansvar gør, at regionen ikke udelukkende fokuserer på tallene, men også vil medtage andre aspekter når en ny teknologi muligvis skal implementeres for at forbedre arbejdsforhold for personalet. 

Samfundet forventer at regionen løfter ansvaret. Således regionen på den måde bidrager til at personalet kan blive i deres arbejdsposition i flere år, og dermed går senere på pension. Et andet aspekt i forhold til regionen er, at regionen ønsker at fremstå godt ud ad til, som en attraktiv region der kan tiltrække arbejdskraft og borgere. 

Det sidste forhold, der er medtaget i denne analyse er forskellene i lønomkostninger. I scenarie 1 skulle der 1,2 sonograf til for at bemande en stue fem dage om ugen. I dette scenarie skal der kun 1 sonograf til. Da det antages, at en sonograf nu kan scanne fem dage om ugen, altså fuldtid, hvorimod sonografen før kun kunne holde til at scanne fire dage om ugen. Dette giver lønomkostninger på årsbasis:
\begin{equation}
323604 \text{ DKK}\cdot1 = 323604 \text{ DKK pr. stue}
\end{equation}
Behovet for færre sonograf til bemanding af én stue, vil sandsynligvis over tid føre til, at færre sonografer skal uddannes. Dette vil føre til en økonomisk gevinst for regionen, da udgifterne til uddannelse af sonografer vil blive nedsat. 

\section{Perspektivering til Regionshospitalet Midt Viborg, Afdeling Kvindesygdomme og Fødsler}
I forbindelse med denne mini-MTV er der også indhentet oplysninger fra RMV gennem et interview. Formålet med interviewet var at få afdækket de samme områder, der blev afdækket ved interview med HEH. 

Set fra et økonomisk perspektiv er omkostningerne til at holde en stue i drift på RMV sammenlignelige med de beskrevne for HEH. Dette viser, at denne omkostnings- og ressourceanalyse er mulig at overføre til lignende afdelinger på andre hospitaler i Danmark. 

\section{Delkonklusion}
Denne gennemgang af omkostnings- og ressourceforskelle mellem scenarie 1 og scenarie 2 viser, at der er en række fordele ved scenarie 1 såvel som scenarie 2. Hvilken der er den afgørende faktor er dermed op til den enkelte potentielle indkøber at afgøre. De største forskelle er ved lønudgifter, penge til forebyggelse og penge til anskaffelse af ultralyds robotarm. 

Hvis der kun kigges på de direkte omkostninger, vil scenarie 1 være den løsning, med mindst omkostninger. Dette skyldes at scenarie 2 er dyrere i direkte omkostninger, da robotarmen er en add-on, og besparelsen på 0,2 sonograf ikke dækker udgiften til robotarmen på 400.000 DKK. \\
Hvis der medtages de indirekte og afledte omkostninger, vil scenarie 2 give mulighed for besparelser for regionen, i forhold til forebyggelse af arbejdsskader og mulige sygedage. Dette kan opveje for udgifterne til robotarmen, men vil naturligvis være afhængig af den enkelte hospitalsafdeling. Scenarie 2 giver mulighed for længere tid på arbejdsmarkedet, hvorved tidlig pension undgås, hvilket giver mindre omkostninger for samfundet.

