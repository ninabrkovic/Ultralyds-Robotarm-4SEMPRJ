\chapter{Resumé}
\textbf{Indledning} \\ 
Sonografer, der arbejder med at scanne gravide, er i risiko for arbejdsgener som følge af akavede arbejdstillinger. Under arbejdet er det ofte nødvendigt at presse ultralydsproben mod maven, mens proben bliver holdt i strakt arm. En Ultralyds Robotarm vil mindske antallet af akavede arbejdsstillinger. Robotarmen, hvorpå en ultralydsprobe er påmonteret, vil blive styret af sonografen via et joystick. Derved undgår sonografen de tidligere nævnte fysiske udfordringer og eventuelle gener. 

\textbf{Metoder} \\ 
Målet har været at undersøge hvilke konsekvenser og følger, implementering af Ultralyds Robotarmen kan  have med henblik på de fire elementer: Teknologi, Organisation, Patient og Økonomi. \\
Under interview med afdelinger for scanninger af gravide på Hospitalsenheden Horsens og Regionshospitalet Midt Viborg er oplysninger, som kan bruges ved samtlige perspektiver, blevet indhentet. Samtidigt er der ved hvert perspektiv blevet benyttet specifikke metoder. 

\textbf{Diskussion/Perspektivering} \\ 
Rapporten bygger på en sammenfatning af interviews, videnskabelige artikler og antagelser. Dette skyldes delvist, at Ultralyds Robotarmen ikke er færdigudviklet, hvilket giver en usikkerhed, som igennem andre undersøgelser vil kunne blive mindsket. \\
Ultralyds Robotarmen vil  i fremtiden kunne benyttes som telemedicinsk udstyr. Dette vil kræve en tilkobling af kamera og mikrofon, men ingen direkte ændringer på systemet. Sonografen behøver derved ikke at være til stede i samme rum som den gravide.   

\textbf{Konklusion} \\
Ultralyds Robotarmen vil kunne reducere mængden af arbejdsgener samt mindske akavede arbejdsstillinger for sonografer under scanninger af gravide. Dermed vil sonograferne kunne foretage scanninger 37 timer om ugen samt blive længere tid i jobbet. Robotarmen vil ikke føre til ændringer for patienten med hensyn til kvalitet og resultat af scanningen. \\
Teknologisk set har Ultralyds Robotarmen en begrænsning, der gør, at den kan benyttes til 70-80\% af scanningerne på gravide. De resterende 20-30\% af scanningerne skal foretages manuelt.\\
Fra et økonomisk perspektiv vil besparelser i lønomkostninger ikke dække den årlige afskrivning på indkøbet af Ultralyds Robotarmen.