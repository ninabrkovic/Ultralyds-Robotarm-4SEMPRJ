\chapter{Resumé}
\textbf{Indledning} \\ 
Sonografer, der arbejder med at scanne gravide, er i risiko for arbejdsskader  som følge af akavet arbejdstillinger. I arbejdet er det ofte nødvendigt at presse ultralydsproben mod maven, mens proben eksempelvis bliver holdt i strakt arm. En Ultralyds Robotarm vil mindske antallet af akvadede arbejdsstillinger. Robotarmen, hvorpå en ultralydsprope er påmonteret, vil blive styret af sonografen via et joystick. Derved undgår sonografen de tidligere nævnte fysiske udfordringer og eventuelle gener. 

\textbf{Metoder} \\ 
Målet har været at undersøge hvilke konsekvenser  og følger implementering af Ultralyds Robotarmen kan  have, med henblik på de fire elementer teknologi, organisation, patient og økonomi. \\
Under interview med Hospitalsenheden Horsens- og Regionalshospitalet Midt Viborg- afdelinger for scanninger af gravide er  oplysninger, som kan bruges ved alle perspektiverne, blevet indhentet. Samtidigt er der ved, hvert perspektiv blevet benyttet  specifikke metoder for hvert emne. 

\textbf{Diskussion/Perspektivering} \\ 
Rapporten bygger på en sammenfatning af interviews, videnskabelig artikler og antagelser. Det skyldes delvist at Ultralyds Robotarmen ikke er færdig udviklet. Dette giver en usikkerhed, som igennem andre undersøgelser vil kunne blive mindsket. \\
Ultralyds Robotarmen vil  i fremtiden kunne benyttes som telemedicinsk udstyr. Dette vil kræve en sikker internetforbindelse, men ikke direkte ændringer på udstyret, som nemt kan deles op i en patient-del og en sonograf-del.  
