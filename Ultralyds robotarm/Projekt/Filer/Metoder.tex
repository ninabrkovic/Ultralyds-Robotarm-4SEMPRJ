\chapter{Metoder}
Afsnittet indeholder en beskrivelse af hvilke metoder, der er blevet anvendt til udarbejdelse af denne mini-MTV i forhold til de fire MTV aspekter: Teknologi, Patient, Organisation og Økonomi.

Overordnet er der blevet gennemført en littetursøgning og -vurdering på baggrund af en i forvejen opstillet protokol (Bilag xx). Protokollen er udarbejdet ud fra specifikke søgestrategier, hvor der er søgt på både engelsk og dansk. De specifikke søgeord er medtaget som dokumentation. Der er søgt i følgende databaser: Embase, PubMed, Google Scholar, Cochrane og Engineering Village. \\
Udover ovenstående litteratur er der, efter behov, søgt efter ikke videnskabelig litteratur for at opnå en forståelse for opbygningen af sonograf uddannelse, ultralyds scanning og andre løse emner for at komme ind i problemstillingen. 

I udvælgelsen af litteratur har der været eksklusions og inklusions kriterier. Eksklusions kriterier har været telemedicin, normal ultralydsscanning af andre patienter end gravide. Inklusions kriterier har været artikler omhandlende ultralydsscanning af gravide (angående deres oplevelse), scanning med robotarm, sonograf arbejdsskader hvor forholdende i det pågældende land er sammenlignelige med Danmark. 

Under projektet blev der taget kontakt til "Kvindeafdelingen, Svangre- og ultralydsambulatorium" på Hospitalsenheden Horsens. Her blev der udført et interview med afdelingssygeplejersken for "svangre- og ultralydambulatorium" Tina Arnbjørn. Derudover blev der foretaget uofficielle interview med tre af sonograferne på afdelingen. Da der kun var 6 sonografer ansat, blev der ikke foretaget et spørgeskema. 

\section{Teknologi}
Den teknologiske dataindsamling er primært sket på baggrund af korrespondance med udvikleren bag Ultralyds Robotarmen, Søren Pallesen, da robotarmen ikke er fuldt færdigudviklet. Vurderingen er udarbejdet på baggrund af denne korrespondance og Søren's antagelser om den færdige robotarm. 
\section{Patient}
Patient dataindsamlingen er sket på baggrund af interview med Tina Arnbjørn. Vurderingen er udarbejdet med udgangspunkt i "Udforskning af de fem patientaspekter i MTV". 
\section{Organisation}
Den organisatoriske dataindsamling er sket på baggrund af videnskabelige artikler og interview med Tina Arnbjørn. 
\section{Økonomi}
Den økonomiske dataindsamling er primært sket på baggrund af direkte kontakt til kilder via telefon eller mailkorrespondance. Derefter er dokumenter og andre skriftlige kilder afsøgt, typisk ved at holde dem op mod mundtlige kilder. Vurderingen er udarbejdet med udgangspunkt i følgende sundheds økonomiske analyser: Omkostningsminimerings analyse (CMA) og cost-effectiveness (CEA). De økonomiske beregninger indeholder flere af projektgruppens antagelser, hvor det ikke har været muligt at finde kilder med tilstrækkelig økonomisk evidens. 