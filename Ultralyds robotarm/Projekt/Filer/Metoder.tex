\chapter{Metoder}
Afsnittet indeholder en beskrivelse af hvilke metoder, der er blevet anvendt til udarbejdelse af denne mini-MTV i forhold til de fire MTV aspekter: Teknologi, Patient, Organisation og Økonomi.

Overordnet set er der blevet gennemført en litteratursøgning og -vurdering på baggrund af en i forvejen opstillet protokol (Bilag xx). Protokollen er udarbejdet ud fra specifikke søgestrategier, hvor der er søgt på både engelsk og dansk. De specifikke søgeord er medtaget som dokumentation. Der er søgt i følgende databaser: Embase, PubMed, Google Scholar, Cochrane og Engineering Village. \\
Udover ovenstående litteratur er der, efter behov søgt efter ikke videnskabelig litteratur for at opnå en forståelse for opbygningen af sonograf uddannelse, ultralydsscanning og andre løse emner for at komme ind i problemstillingen. 
Evt. skriv noget om inklusion og eksklusion kriterier
Eksklusion:
\begin{itemize}
\item Telemedicin
\item Scanning af hjerte mv.
\end{itemize}
Inklusion:
\begin{itemize}
\item Scanning af gravide
\item Robotarm
\item Arbejdsskader, hvor forhold er sammenlignelige med DK
\end{itemize}

\section{Teknologi}
\section{Patient}
Patientafsnittet bygger på data fra interview og samtale med sonografer på HEH, samt viden fra etik undervisning. Derudover er de fem patientperspektiver – sociale, økonomiske, etiske, individuelle og kommunikative forhold – blevet benyttet til at belyse den pågældende teknologi og de faktorer, der har betydning for patientens hverdagsliv.\\ 
Den etiske vurdering tager udgangspunkt i problemstillinger, der påvirker gravide og sonografer. Disse problemstillinger omhandler de professionsetiske principper, som den etiske vurdering er baseret på.
\section{Organisation}
Dataindsamling til Organisationsafsnittet er sket via direkte kontakt til kilder gennem interviews i både HEH og RMV. Afsnittet belyser betydningen af implementeringen af den nye teknologi for afdelingen som organisation samt mulige ændringer for personalet. Derudover er Leavitts organisationsmodel blevet benyttet til at beskrive de fire organisatoriske hovedelementer.

\section{Økonomi}
Den økonomiske dataindsamling er primært sket på baggrund af direkte kontakt til kilder via telefon eller mailkorrespondance. Derefter er dokumenter og andre skriftlige kilder afsøgt, typisk ved at holde dem op mod mundtlige kilder. Vurderingen er udarbejdet med udgangspunkt i følgende sundheds økonomiske analyser: Omkostningsminimeringsanalyse (CMA) og cost-effectiveness (CEA). De økonomiske beregninger indeholder flere af projektgruppens antagelser, hvor det ikke har været muligt at finde kilder med tilstrækkelig økonomisk evidens. 