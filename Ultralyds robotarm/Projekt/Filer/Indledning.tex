\chapter{Indledning} 
Ved ultralydsscanning af de gravide holder sonograferne proben i akavede stillinger og skal presse med ca. 10-15 kg for at få et klart billede. Disse stillinger øger muligheden for at få arbejdsskader. Der sker desuden en yderligere belastning, da der er en stigning i antallet af overvægtige \cite{Overvaegt}. Dette betyder at sonograferne skal presse med en større kraft for at få klare billeder. Grundet sonografernes arbejdsstillinger, er der fra Føtalmedicinsk Selskab kommet guidelines angående det maksimale antal af timer, en sonograf må foretage scanninger i løbet af en uge. Dette gør at der skal flere sonografer til for at kunne scanne det stigende antal gravide \cite{Foedsler}.

Denne udvikling har ført til, at der er blevet udviklet en ultralydsrobotarm. Denne robotarm styres via et joy-stick, således sonograferne ikke skal være i akavede stillinger, men istedet kan styre robotten til de ønskede stillinger. 

Formålet med denne rapport er at undersøge om en ultralydsrobotarm vil kunne gøre det lettere at håndtere den stigende mængde gravide og samtidig mindske chancen for arbejdsskader. 

\section{Baggrund}
Projektet er lavet på baggrund af udviklingen af ultralydsrobotarmen. Der ønskes at finde frem til om denne robotarm kan erstatte noget af det eksisterende udstyr og derved give en bedre effekt end det gamle udstyr.

\section{Projektafgrænsning}
 I projektet er der valgt at fokuserer på ultralydsrobotarmen som en mulighed for at mindske arbejdsskaderne for sonograferne. Der er fravalgt at kigge på den telemedicinske del af robotarmen, da denne del ikke var færdig udviklet i det tidsrum hvor Mini-MTV'en er blevet udarbejdet.  
 Der blevet forsøgt taget kontakt til både "Kvindeafdelingen, Svangre- og ultralydsambulatorium" på Hospitalsenheden Horsens og afdelingen "Kvindesygdomme og Fødsler" på Aarhus Universitetshospital Skejby. Disse blev valgt for at kunne udfører interview angående den daglige gang for sonografer på en afdeling. Grundet tidsbegrænsningen blev der kun taget kontakt til to afdelinger på hver sit sygehus. 

\label{version_Systemark}
%\end{longtabu}