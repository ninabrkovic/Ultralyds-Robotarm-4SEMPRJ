\chapter{Indledning} 
Ved ultralydsscanning af de gravide holder sonograferne proben i akavede stillinger og skal presse med ca. 3 kg for at få et klart billede. Ved kompliceret scanninger skal der trykkes med op til 11 kg. REFERENCE TIL SØREN. Disse stillinger øger risiko for at få arbejdsskader. Der sker desuden en yderligere belastning, da der er en stigning i antallet af overvægtige \cite{Overvaegt}. Dette betyder at sonograferne skal presse med en større kraft for at få klare billeder. Grundet sonografernes arbejdsstillinger, er der fra Dansk Føtalmedicinsk Selskab (REFERENCE til bilag) kommet guidelines angående det maksimale antal af timer, en sonograf må foretage scanninger i løbet af en uge. Dette gør at der skal flere sonografer til for at kunne scanne det stigende antal gravide \cite{Foedsler}.

Denne udvikling har ført til, at der er blevet udviklet en Ultralyds Robotarm. Denne robotarm styres via et joystick, således sonograferne ikke skal være i akavede stillinger, men i stedet kan styre robotten til de ønskede stillinger. 
Billede af opstilling - evt. på forsiden.  

Formålet med denne rapport er at undersøge om en Ultralyds Robotarm vil kunne gøre det lettere at håndtere den stigende mængde gravide og samtidig mindske chancen for arbejdsskader.  

\section{Baggrund}
Projektet er lavet på baggrund af udviklingen af Ultralyds Robotarmen. Der ønskes at finde frem til om denne robotarm kan erstatte noget af det eksisterende udstyr og derved give en bedre effekt end det gamle udstyr. - SKAL OMFORMULERES 


\section{Projektafgrænsning}
 I projektet er der valgt at fokusere på Ultralyds Robotarmen som en mulighed for at mindske arbejdsskaderne for sonograferne. Der er fravalgt at vurdere på den telemedicinske del af robotarmen, da denne del ikke var færdig udviklet i det tidsrum hvor mini-MTV'en er blevet udarbejdet. HVAD ER FORMÅLET MED MTV'en, fokus spørgsmål og overordnet mtv spørgsmål. 
 Der blev taget kontakt til både "Kvindeafdelingen, Svangre- og Ultralydsambulatorium" på Hospitalsenheden Horsens (HEH) og afdelingen "Kvindesygdomme og Fødsler" på Regionshospitalet Viborg (RMV). Disse er valgt for at kunne udføre interview angående den daglige gang for sonografer på en afdeling. Grundet tidsbegrænsningen blev der kun taget kontakt til to afdelinger på hver sit sygehus. 

\label{version_Systemark}
%\end{longtabu}