\chapter{Teknologi} \label{Teknologi}
I dette afsnit undersøges Ultralyds Robotarmen ud fra et teknologisk perspektiv.  \\
Den teknologiske løsning består af:
\begin{itemize}
\item UR3 Robotarm fra Universal Robots incl. software til styring af denne
\item Stativ til robotarmen
\item Joystick med dummy-probe
\item Computer
\item Universalholder til ultralydsprobe
\end{itemize}
Denne løsning skal kobles til det allerede eksisterende udstyr. Derfor er produktet en add-on løsning, hvilket betyder, at produktet skal købes udover det eksisterende scanningsudstyr. Det nuværende system består af GE Voluson S6 inkl. DICOM (Digital Imaging and Communications in Medicine) og en printer, samt diverse ultralydsprober, se Bilag 4 og Bilag 5. GE Voluson S6, som ses på figur \ref{udstyrHorsens}, omtales som arbejdsstationen. 

%Se bilag økonomi. 
\begin{figure}[H]
	\begin{minipage}{0.45\textwidth}
		\centering
		\includegraphics[width=\textwidth]{Figurer/udstyrHorsens.jpg}
		\caption{Eksisterende udstyr: GE Voluson S6 med tilbehør. Billede taget på HEH.}
		\label{udstyrHorsens}
	\end{minipage}
	\hspace{0.02\textwidth}
	\begin{minipage}{0.58\textwidth}
		\centering
		\includegraphics[width=\textwidth]{Figurer/StativMedUR3Render.png}
		\caption{Grafisk model af robotarmen med stativ.}
		\label{Robotstativ}
	\end{minipage}
\end{figure}

\section{Anvendelsesområde}
Produktet skal anvendes til ultralydsscanning af gravide. Robotarmen er fastmonteret på et stativ, så den kan hænge over den gravide. Dette medvirker til en større trykkraft, end hvis stativet stod på gulvet. For at skabe mobilitet, er stativet placeret på hjul. Af sikkerhedsmæssige årsager kan hjulene låses. Dette giver mulighed for en statisk placering i forhold til den gravide. På robotarmen findes en universalholder til ultralydsproben. Holderen passer til alle større fabrikanters håndholdte ultralydsprober, bortset fra vaginalprober, som ikke kan benyttes med robotarmen, se Bilag 12, 28.04.2016.

Stativet med robotarmen skal være på modsatte side af sengen end sonografen. Dette sikrer sonografens udsyn og kontakt til den gravide. \\
Robotarmen holder ultralydsproben over den gravide, mens den styres af sonografen via et joystick. Derved undgår sonografen akavede arbejdsstillinger.

Joysticket har en påmonteret dummy-probe, som ikke har probeegenskaber. Dummy-proben giver sonografen en følelse af at sidde med en ægte probe i hånden.
Systemet skal kunne overføre det tryk, som sonografen påvirker joysticket med til robotarmen, se Bilag 12, 28.04.2016. Derved udføres tryk under scanningen på samme vilkår, som hvis sonografen trykkede direkte på den gravide.
 

\begin{figure}[H]\centering
	\includegraphics[width = 1.0\textwidth]{Figurer/ergonomiskLosning.jpg}
	\caption{Eksempel på opstilling af Ultralyds Robotarm. Her er robotarmen placeret over patienten. På billedet ses joystick (tv.) og robotarm (th.).  }
	\label{ergonomiskLosning}
\end{figure}

\newpage
\section{Specifikationer}
Robotarmen har en rækkevidde på 50 cm, der angiver hvor langt den kan række ud, som var det en arm. Den vejer 11 kg, og stativet skal derfor være bygget dertil. Robotarmen har 6-graders frihed, som betyder at den kan bevæge sig i x-, y- og z-aksens retning med drejevirkning om hver akse, se figur \ref{seksgradersfrihed}. Samtidigt kan den lave en +/- 360 graders rotation. \\
Robotarmen kræver en 100-240 VAC, 50-60 Hz strømforsyning, hvilket betyder at den kan blive sat til en almindelig dansk stikkontakt, se Bilag 1.
\begin{figure}[H]\centering
	\includegraphics[width = 0.3\textwidth]{Figurer/sixDegressOfFreedom.jpg}
	\caption{Mulige retninger ved 6-graders frihed. Bevægelse i x-, y- og z-aksens retning og drejevirkning om hver akse \cite{6gradersfrihed}. }
	\label{seksgradersfrihed}
\end{figure}
Joysticket har bevægelighed som et håndled, og derved har den også de begrænsninger, som findes ved et håndled. Det har 6-graders frihed, se figur \ref{seksgradersfrihed}, hvilket passer med robotarmen, se Bilag 3. 

Der medfølger software til Ultralyds Robotarmen, hvori styringen af robotarmen og joystick ligger. Det er her bevægelserne fra joysticket omsættes til robotarmens bevægelser. Her er flere sikkerhedsmæssige foranstaltninger placeret. Nogle er indbygget i robotarmen, eksempelvis stopper den øjeblikkeligt, hvis den bliver mødt af en kraft på 50 N (ca. 5 kg) eller derover, se Bilag 2.    

\section{Effektivitet}
Det antages, at Ultralyds Robotarmen vil blive benyttet til 70-80\% af scanningerne af gravide, da de sidste 20-30\% af scanningerne er for komplicerede til, at systemet kan udføre dem. Derfor skal sonografen manuelt foretage de sidste 20-30\% af scanningerne med den nuværende metode, se Bilag 12, 28.04.2016. De komplicerede ultralydsscanninger er blandt andet scanninger på kvinder med højt BMI eller kvinder med bagoverbøjet livmoder, se Bilag 12, 28.04.2016. 
 
Ultralyds Robotarmen har ikke indflydelse på billedkvalitet eller resultatet af scanningen. Dette kommer af, at ultralydsproberne er de samme, som man før har benyttet. Computeren i Voluson S6 indeholder det samme software til billedanalyse og til diverse instrumenter, som bruges under nuværende scanninger. Der er blandet andet tale om software til vækst- og flowmålinger af fostre. \\
Når sonografen vil trykke med ultralydsproben på den gravide, vil trykkraften blive overført til joysticket, så sonografen får den korrekte trykfeedback. Derved vil sonografen have følelsen af, at der bliver trykket direkte på den gravide og kan dermed bedre selv have føling med situationen, se Bilag 6. 

\section{Sikkerhed}
I softwaren til styring af robotarmen findes en sikkerhedsindstilling, hvor en grænse for trykpåvirkningen fra joysticket skal indstilles. Hvis der, af menneskelige eller tekniske fejl, bliver påvirket med en kraft over grænsen, vil robotarmen automatisk slå fra og stoppe, se Bilag 12, 28.04.2016. \\
Som tidligere nævnt er der også en sikkerhed i, at robotarmen stopper sine bevægelser, hvis den bliver ramt af en kraft på over 50 N, se Bilag 2. Denne foranstaltning gør, at den ikke vil gøre skade på mennesker eller genstande ved at ramme dem, samt at der ikke er behov for et eventuelt sikkerhedsgitter. 

\section{Delkonklusion}
Ultralyds Robotarmen kommer til at udføre 70-80\% af scanningerne, hvilket gør at sonograferne skal udføre nogle af scanningerne manuelt. En stor del af belastningen på sonograferne fjernes, når robotarmen udfører størstedelen af scanningerne. Sonograferne vil potentielt have mere styrke til at udføre de mest komplicerede scanninger manuelt. \\
Ved implementering af Ultralyds Robotarmen vil resultatet af scanningen ikke blive påvirket, da der stadig er samme bevægelsesfrihed for ultralydsproben. Yderligere skaber de forskellige sikkerhedsindstillinger en tryghed ved brugen af systemet.\\ 
Da produktet er en add-on løsning vil styring, betjening og implementering af Ultralyds Robotarmen ikke give de store problemer. Styringen og betjeningen vil gøres lettere gennem brugen af dummy-proben.  
Der vil naturligvis altid være en overgangsperiode, hvor sonograferne skal vænne sig til at benytte teknologien. 


