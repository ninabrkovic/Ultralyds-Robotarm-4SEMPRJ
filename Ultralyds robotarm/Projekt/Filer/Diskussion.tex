\chapter{Diskussion}
%Større økonomisk analyse
%Kvalitets diskussion - Er den virkelig lige så god ved brug af robotten
%Hvad ville have overbevidst os hundred procent til 
Denne mini-MTV rapport bygger bl.a. på data opsamlet igennem interviews fra to af landets ultralydsscanningsafdelinger, HMH og RMV, se Bilag 4 og Bilag 5. Dette gør det svært at generalisere for samtlige sygehuse og hospitaler på landsplan. Interviews med flere ultralydsscanningsafdelinger ville derfor give et mere klart billede af, hvordan situationen er på de forskellige afdelinger. Ved implementering af Ultralyds Robotarmen kan det også være meget nyttigt at kende til de enkelte afdelingers strukturer. \\
Idet svarene fra de to afdelinger er relativ ens, vil man med en god tilnærmelse kunne drage nogle paralleller til den overordnede opbygning og struktur på ultralydsafdelingerne i Danmark.

En større undersøgelse, som skal vise antal og baggrunden for sonografernes sygemeldinger, ville være et af de næste skridt, der skulle foretages. Det har ikke været muligt at få data på sygefravær i forbindelse med arbejdsrelaterede gener fra de to afdelinger. En sådan undersøgelse kunne derfor være med til at underbygge behovet for implementering af robotarmen. \\
Efter samtale med sonograferne på afdelingen er holdningen dog, at arbejdet er hårdt fysisk belastende.  Sonograferne vil derfor ikke kunne holde til arbejdet som sonograf indtil pensionistalderen, med mindre der sker nogle ændringer. Selvom de er meget glade for jobbet, vil de kunne blive nødsaget til at skifte karriere eller stilling, se Bilag 4. 

% Sonograferne aflastes af Ultralyds Robotarmen, men der vil endnu være behov for ergonomiske værktøjer og redskaber. Scanningerne foretages med samme fremgangsmåde som før, dog er arbejdet fjernet fra strakt arm over den gravide, til arbejde med dummy-proben. Dette gør at sonografen ikke har samme akavede arbejdsstillinger som tidligere. 

Et teststudie vil være en metode, hvormed det kan testes om denne løsning vil kunne fungere på landets afdelinger. Her skal systemet implementeres på en afdeling, hvor brugen følges tæt. Dette studie vil derfor kunne benyttes til at verificere Ultralyds Robotarmens effekt. Sådan et studie har der dog ikke været mulighed for at udføre, da robotarmen endnu ikke er færdigudviklet. \\
Afdelingerne som blev interviewet er begge meget åbne overfor teknologien. HEH har indvilliget i at teste robotarmen, når denne er blevet færdigudviklet, og RMV vil gerne være med i et eventuelt fremtidigt samarbejde, se Bilag 4 og Bilag 5. 

Stigningen i kvinders BMI kan have en betydning for robotarmens effekt \cite{kvinderovervaegt}. Disse scanninger er mere komplicerede og udgører cirka 20\% af det samlede antal scanninger. De komplicerede scanninger skal udføres manuelt. Idet BMI’en forøges, bliver denne procentdel større, og derfor skal flere og flere scanninger foretages manuelt, se Bilag 12, 28.04.2016. 

