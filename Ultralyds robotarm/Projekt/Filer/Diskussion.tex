\chapter{Diskussion}
Denne mini-MTV bygger blandt andet på data opsamlet igennem interviews fra to af landets ultralydsscannings afdelinger HEH og RMV, se Bilag 4 og Bilag 5, hvilket gør det svært at generalisere resultaterne for samtlige hospitaler på landsplan. Interviews med flere ultralydsscanningsafdelinger ville give et mere klart billede af, hvordan situationen er på andre lignende afdelinger. Ved implementering af Ultralyds Robotarmen vil det være nyttigt at kende til de enkelte afdelingers strukturer. \\
Idet svarene fra de to afdelinger er relativ ens, vil man med en god tilnærmelse kunne drage nogle paralleller til den overordnede organisering og struktur på ultralydsscannings afdelingerne i Danmark.

En videregående undersøgelse, som skal vise antal og baggrunden for sonografernes sygemeldinger i Danmark, vil med fordel kunne udarbejdes. Det har ikke været muligt at indsamle data fra de to afdelinger omhandlende sygefravær forårsaget af arbejdsgener. En sådan undersøgelse kunne derfor være med til at underbygge behovet for implementering af robotarmen. \\
Efter samtale med sonograferne på afdelingerne er holdningen, at arbejdet er fysisk hårdt belastende.  Sonograferne vil derfor ikke kunne holde til arbejdet som sonograf indtil pensionsalderen, medmindre arbejdsmiljøet forbedres. Selvom de er meget glade for jobbet, vil de kunne blive nødsaget til at skifte karriere eller stilling, se Bilag 4. 

Et teststudie vil være en metode, hvormed det testes om Ultralyds Robotarmen vil kunne fungere på landets afdelinger. Her skal systemet implementeres på en afdeling, hvor brugen følges tæt. Dette studie vil kunne benyttes til at verificere Ultralyds Robotarmens effekt. Et teststudie har ikke været mulighed for at udføre, da robotarmen endnu ikke er færdigudviklet. \\
De interviewede afdelinger er begge åbne over for teknologien. HEH har indvilliget i at teste Ultralyds Robotarmen, når den er færdigudviklet, og RMV vil gerne være med i et eventuelt fremtidigt samarbejde, se Bilag 4 og Bilag 5. 

En udvidet økonomisk analyse vil give et forbedret økonomisk evidensgrundlag for implementeringen af Ultralyds Robotarmen, da der medtages flere økonomiske aspekter, der gælder på landsplan. Her skal medtages udgifter til forebyggelse af arbejdsgener hos sonograferne. Denne analyse vil ikke være relevant, før robotarmen er færdigudviklet og testet. Udgifter til service af robotarmen skal medtages og holdes op imod de udgifter, der er til service af det eksisterende udstyr. Der skal endvidere testes, om robotarmen har en længere levetid end det eksisterende ultralydsudstyr. Dette vil medvirke til en længere afskrivningsperiode end de angivede ti år, se Kapitel \ref{Okonomi}. 

Det er en antagelse, at kvaliteten vil være af samme standard som med det eksisterende ultralydsudstyr, da Ultralyds Robotarmen ikke er færdigudviklet. Denne antagelse underbygges af videnskabelige artikler, der har testet kvaliteten af ultralydsscanninger med en robotarm \cite{8}\cite{18}\cite{Hjerterobot}. Disse videnskabelige artikler beskriver udstyr, som er sammenlignelige med Ultralyd Robotarmen. Konklusionen i begge videnskabelige artikler er, at kvaliteten er den samme som ved den nuværende scanningsmetode. Med brugen af robotarmen tager det i gennemsnit længere tid at scanne, for at opnå samme kvalitet som det eksisterende udstyr \cite{18}. 

Sonograferne aflastes af Ultralyds Robotarmen, men der vil stadig være behov for ergonomiske redskaber. Scanningerne foretages med samme fremgangsmåde som tidligere, men arbejdet er fjernet fra strakt arm til arbejde med joystick, se afsnit \ref{aktoerer_organisation}. Sonograferne skal selv udføre de komplicerede scanninger, hvor de bliver belastet i akavede arbejdsstillinger. Antallet af komplicerede scanninger er mindre end de konventionelle scanninger, så den samlede belastning vil være mindre end uden Ultralyds Robotarmen \cite{24}. 

Stigningen i kvinders BMI kan have en betydning for robotarmens effekt \cite{kvinderovervaegt}. Scanninger af gravide med højt BMI er ofte mere komplicerede og udgør 20-30\% af det samlede antal scanninger, se Bilag 12, 28.04.2016. De komplicerede scanninger skal udføres manuelt. Idet BMI’en forøges, bliver andelen af komplicerede scanninger større, og derfor skal flere scanninger foretages manuelt \cite{24}\cite{31}\cite{8}. 

Ultralyds Robotarmen er ikke færdigudviklet, derfor har det ikke været muligt at teste robotarmens funktioner på en ultralydsscannings stue. En test ville kunne vise, hvor stor en indflydelse Ultralyds Robotarmen har på arbejdsgener, samt hvor hurtigt sonografer lærer at bruge joysticket. \\ 
Et færdigudviklet produkt vil være lettere at analysere angående pris og tekniske specifikationer. Prisen er et vigtigt aspekt, da den har indflydelse på, om det økonomisk kan betale sig at anskaffe Ultralyds Robotarmen. 