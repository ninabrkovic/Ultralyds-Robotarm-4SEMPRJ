\chapter{Organisation}
Dette afsnit vil give et indblik i strukturen og opbygningen af "Kvindeafdelingen, Svangre- og ultralydsambulatorium" på Hospitalsenheden Horsens og "Kvindesygdomme og fødsler" på Hospitalsenheden Midt i Viborg. Afsnittet vil belyse, hvilken betydning implementeringen af en ultralyds robotarm, vil have for afdelingen som organisation, samt hvilke ændringer dette vil medføre i arbejdsgangen for personalet. \\
Informationer, som er indhentet fra afdelingen på Hospitalsenheden Horsens og Hospitalsenheden Midt, vil blive sammenholdt med videnskabelige artikler, i forsøget på at finde en større sammenhæng i problemstillingen omkring arbejdsgener ved ultralydsscannings arbejdet.

Det er valgt, at benyttes Leavitts organisations model \ref{LeavittModel}. Denne model er en diamantmodel, der arbejder med fire organisatoriske hovedelementer, der relaterer sig til hinanden. Hvert hovedelement vil blive belyst i hvert sit underafsnit. 

\begin{figure}[h!]\centering
	\includegraphics[width = 0.5\textwidth]{Figurer/LeavittModel}
	\caption{Leavitts organisations model, viser hvordan struktur, aktører, opgaver og teknologi indbyrdes relaterer sig til hinanden, i midten haves kulturen for organisationen.}
	\label{LeavittModel}
\end{figure}
I analysen er der kun medtaget to ultralydsafdeling, og derfor er der ikke videre empiri for at kunne drage konklusioner om at billedet vil være det samme på andre lignende hospitals afdelinger i Danmark. 

TYDELIGGØR AT INFORMATIONER ER INDHENTET GENNEM INTERVIEW med Tina Arnbjørn og de tre sonografer og Dem fra Viborg (Tove og hende den anden), ARTIKLER MV.

\section{Kvindeafdelingen, Svangre- og ultralydsambulatorium, Hospitalsenheden Horsens}
Afdelingen på Horsens er bemandet af 1 afdelingssygeplejerske, 5 sonografer samt et ukendt antal læger. Antallet af læger er ikke relevant for denne analyse, da der udelukkende fokuseres på sonografernes arbejdsgange. Afdelingen har udstyr til fire stuer, hvoraf tre stuer bemandes af sonografer. Der foretages 30-40 scanninger om dagen på afdelingen, hver scanning tager i gennemsnit 35 minutter.

\section{Kvindesygdomme og fødsler, Hospitalsenheden Midt, Viborg}

\section{Opgaver}
Opgaverne som afdelingen i Horsens varetager på nuværende tidspunkt, vil ikke ændrer sig ved implementering af robotarmen, da behovet for scanninger af gravide i Horsens forbliver uændret. Opgaverne består af nakkefoldsscanning i 11.-13. uge, misdannelsesscanning i 19.-22. uge, vægtscanninger samt andre kontrolscanninger i løbet af graviditeten(reference). 

\section{Teknologi}
Ved implementering af ny teknologi, som ultralyds robotarmen, vil det sætte krav til aktørernes faglige kundskaber og erfaringer i brugen af teknologien. Dette er gældende for samtlige sonografer. Derfor vil der skulle være en indkørsels periode af teknologien førend, at den vil være i fuld brug og alt personale har den rette kendskab i brugen af robotarmen. \\
Det vurderes, at de eksisterende stuer på afdelingen i Horsens og i Viborg er tilstrækkelig store til at teknologien vil kunne implementeres uden yderligere ændringer.

\section{Struktur}
På nuværende tidspunkt er den strukturelle opbygning på afdelingen i Horsens, at en medarbejder ultralydsscanner 4 ud af 5 arbejdsdage på en uge. Den femte dag er en aflastnings dag for den enkelte medarbejder, da det er et kendt problem på afdelingen i Horsens at scanningsarbejde er fysisk belastende for medarbejderen. I løbet af en scanningsdag har en medarbejder i gennemsnit ti scanninger. Yderligere foretages der på afdelingen forebyggende tiltag, i form af styrketrænende elastikøvelser, ergonomiske redskaber samt fri adgang til wellness konsulenter, der kontrollerer og vejleder om medarbejderens arbejdsstillinger. (reference)

Implementering af robotarmen vil føre til en ændring i afdelingens strukturelle opbygning for afdelingen i Horsens. Da robotarmen vil kunne gøre scanningsarbejdet væsentlig mindre belastende (reference), vil en medarbejder kunne scanne 5 ud af 5 arbejdsdage om ugen.  

\section{Aktører}
Implementeringen af robotarmen vil føre til markante ændringer for den enkelte sonografs arbejde.

BMI-problemer (ref. statistik), akavede arbejdsstilling (ref. artikler), fysisk belastende, ikke fuldføre arbejdet over tid (ref. udsagn), dedikerede i jobbet - gemmer arbejdsgener væk (ref. artikler + udsagn).

\section{Kultur}
Kulturen på afdelingen i Horsens er meget teknologivenlig. Derfor formodes det, at implementeringen af teknologien ikke vil føre til væsentlige problemer i forhold til at få personalet til at benyttes den nye teknologi. Dog kræves det, at der tilrettelægges en ordentlig plan for oplæring af personalet i brugen af teknologien. 

Afdelingen i Horsens har allerede på nuværende tidspunkt indvilliget i at være testafdeling for Robotic Ultrasound ApS under udviklingen af produktet. Det er i afdelingen interesse, da de ser en fremtid i produktet og dermed ønsker at være med til at tilpasse produktet til afdelingens struktur og behov.   

\section{Delkonklusion}

