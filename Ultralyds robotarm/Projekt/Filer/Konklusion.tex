\chapter{Konklusion}
Potentialet bag Ultralyds Robotarmen ligger i, at den vil kunne mindske mængden af arbejdsgener. Det gøres igennem en mindskelse af akavede arbejdsstillinger ved scanninger er gravide.
De primære forskelle for den enkelte sonograf er, at denne vil kunne foretage scanninger 37 timer om ugen, samt at blive i jobbet indtil lysten stopper og ikke som tidligere, hvor gener kunne tvinge sonografen til skifte job eller gå på tidlig pension. Dette er i tråd med, at teknologi vinder mere plads i det danske sundhedsvæsen.\\
For de gravide patienter medfører brugen Ultralyds Robotarmen ikke ændringer af betydning for resultatet og kvaliteten af scanningen. \\
Ultralyds Robotarmen har en teknologisk begrænsning, som gør at den kan benyttes ved 70-80\% af scanningerne på gravide. Begrænsningen afhænger af, hvor kompliceret scanningerne er, hvilket kan komme af overvægt eller en bagoverbøjet livmoder. \\
Det økonomiske aspekt viser, at den umiddelbare besparelse i lønomkostninger ikke vil kunne dække den årlige afskrivelse på indkøbet af add-on løsningen Ultralyds Robotarm. Derved er det økonomiske aspekter fra forbedret arbejdsmiljø og personaleforhold, som skal opveje indkøbet af Ultralyds Robotarmen.\\  
Da produktet stadig er under udvikling, kan det fører til ændringer som ikke er medtage i denne rapport. Derfor er det nødvendigt at vurderer om det endelige produkt stemmer overens med rapportens udgangspunkter, før rapporten bruges som et beslutningsgrundlag i forhold til implementering af slutproduktet.

Denne rapport har begrænsninger i forhold til, at det ikke har været muligt at finde frem til omkostninger ved arbejdsskader inde for denne branche. 
Samtidigt har det været nødvendigt at lave diverse antagelse, da produktet ikke er færdigudviklet. Det er valgt at se bort fra prisen på den daglige drift i begge økonomiske situationer. Det er valgt, da det anses for at indebærer et større arbejde med at finde frem til priserne, og derved til at være for komplekst til denne rapport.   