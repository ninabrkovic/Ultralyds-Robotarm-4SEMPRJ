\chapter{Patient}
Ved implementering af en ny teknologi, herunder en Ultralyds Robotarm, har det en indvirkning på patienten. Derfor er det vigtig at belyse, hvilken effekt den nye teknologi har på patientgruppen. \\
I denne mini-MTV vil både gravide og sonografer blive placeret i rollen som patienter - gravide da de får foretaget en ultralydsscanning, og sonografer da de er i risiko for arbejdsskader. Begge grupper vil derfor blive belyst i dette afsnit.  

For at kunne udarbejde en fyldestgørende analyse af patientperspektivet er det nødvendigt at belyse flere forhold. Se figur \ref{patientMTV}, hvor de fem patientperspektiver er vist. 
\begin{figure}[h!]\centering
	\includegraphics[width = 1.0\textwidth]{Figurer/PatientaspekterMTV}
	\caption{Udforskning af patientaspekter i MTV, som har betydning for patientens hverdagsliv. \cite{Leavitt}}
	\label{patientMTV}
\end{figure}

I realiteten kan man ikke skille de fem perspektiver fra hinanden, men for analysens skyld er det valgt at dele dem op. Modellen skal ses som patientens samlede oplevelse af den givne teknologi, hvor den inddrager patientens egne erfaringer. 

De fem perspektiver vil blive gennemgået i de følgende underafsnit. 

\section{Sociale forhold}
På nuværende tidspunkt findes skepsis blandt sonografer for, om de kan forsætte med at scanne indtil pensionsalderen. Der er lavet undersøgelser, som viser at 20 \% kommer på førtidspension grundet deres arbejde \cite{32}.  \\ 
For sonograferne vil eventuelle færre arbejdsskader betyde bedre fysiske funktioner i forhold til arbejde og fritid. Dette kan forlænge tiden på arbejdsmarkedet og forbedre personalemiljøet.      

\section{Kommunikative forhold}
Produktet af scanningen, eksempelvis billeder og kønsbestemmelse, vil ikke blive påvirket af Ultralyds Robotarmen. Da Ultralyds Robotarmen er en add-on løsning, vil den ikke ændre på billedudstyret, se Bilag 6. 
For sonograferne vil det kræve en anden introduktion, da de ikke længere vil have fysisk kontakt med den gravide, dog er de stadig placeret i samme rum.  \\
Sonograferne vil igennem bedre arbejdsstillinger potentielt opleve et andet overskud til arbejdssituationen og patientkontakten.  

\section{Individuelle forhold}
Enkelte gravide patienter kan opleve en utryghed ved at få en fremmed teknologi fysisk tæt på sig. Dette kan forøge en eventuel utryghed omkring sikkerheden ved en ultralydsscanning, den gravide i forvejen kan have i første trimester \cite{29}. \\
Sonografernes autoritet vil være en stor tryghedsfaktor for patienten. Derved vil en eventuel utryghed fra den gravide patient blive mindsket, når sonograferne udviser sikkerhed og åbenhed for teknologien. Titlen sonograf giver en vis form for troværdighed, der skaber tillid mellem patient og sonograf, samt sonografens arbejde.\\
Der vil altid være en sonograf til stede under en scanning, hvilket skaber en menneskelig kontakt og en professionel tryghed.\\
For den gravide er det især vigtigt at knytte bånd til fosteret. Derved mener sonograferne på RMV, at så længe de gravide og deres pårørende kan følge med i scanningen på en skærm, vil den nye teknologi ikke have stor påvirkning på den gravide. Se Bilag 5. \\  
Sonograferne er meget engagerede i deres arbejde, hvilket kan være en af grundene til, at de ikke indmelder skader. Dette er på trods af videnskabelige undersøgelser, som viser at mange sonografer døjer med smerter \cite{32}. 
Hvis akavede og fysisk udfordrende arbejdsstillinger for sonograferne undgås, kan det muligvis skabe en bedre opmærksomhed mod den gravide patient - eksempelvis overskud til forklaring af billeder og patientens velbefindende.  

\section{Etiske forhold}
Brugen af en Ultralyds Robotarm danner grundlag for en række etiske problemstillinger, som påvirker både gravide og sonografer. 
Problemstillingerne omhandler de professionsetiske principper \cite{Husted}. Det er valgt at udvælge de vigtigste aspekter under pligt, konsekvenser og idealler: 
\begin{itemize}
		\item Undgå skade af brugeren
		\item Forebygge sygdom og sygelighed og fremme sundhed eller status quo
		\item Lindre lidelse, fremmedgørelse og ubehag
		\item Handle med forståelse og empati
		\item Handle med etisk ansvarlighed overfor sonografer 
\end{itemize} 

\textbf{Undgå skade af brugeren:} \\
Ultralyds Robotarmen skal hverken være til skade for gravide eller sonograferne.

\textbf{Forebygge sygdom og sygelighed og fremme sundhed eller status quo:} \\
Hvis man ud fra et nytteetisk perspektiv kan få flere gravide igennem en scanning på kortere tid og samtidig mindske antallet af arbejdsskader for sonografer, vil ressourcerne blive udnyttet bedst muligt og derved komme flest mulige til gavn. Dette følger de socialetiske ideer i nytteetikken, som ud fra en overordnet forestilling ønsker at fremme nytte og retfærdighed for de mange.
    
\textbf{Lindre lidelse, fremmedgørelse og ubehag:}\\
Ultralyds Robotarmen skal opfylde dette overfor både sonografer og patienten. Det kan tænkes, at patienten kan føle sig fremstillet som et objekt, fordi teknologien kommer tættere på patienten, mens sonograferne kommer længere væk. Dog er sonografen til stede i samme rum som patienten, derved er der stadig en form for menneskelig kontakt. Det kan tænkes, at denne kontakt vil mindske risikoen for fremmedgørelse og ubehag for patienten.   

\textbf{Handle med forståelse og empati:}\\
Ud fra patientens perspektiv kan det opfattes som en ændring af nærhed- og omsorgsrelationen mellem patienten og sonografen under en scanning med Ultralyds Robotarmen. 

\textbf{Handle med etisk ansvarlighed overfor sonografer:}\\
En af Ultralyds Robotarmens hovedfunktioner er at mindske antallet af sonografernes arbejdsskader. Derved skabes der empati for sonografernes arbejdssituationen.\\
Resultatet er, at en mindskelse i antallet af arbejdsskader vil fremme personalesikkerhed og -trivsel.   		

\section{Økonomiske forhold}
Ultralyds Robotarmen kommer ikke til at have økonomisk indvirkning for den gravide patient. Derimod ligger betalingen og andre tilkoblede ydelser hos den pågældende afdeling og dens ledelse. Dette uddybes i afsnittet Økonomi \ref{Okonomi}. \\
For sonograferne vil Ultralyds Robotarmen heller ikke have en økonomisk indvirkning. Deres løn og timeantal vil forblive ens. 
Både på HEH og RMV ser sonograferne fordele ved Ultralyds Robotarmen, dog menes det, at økonomien og ledelsens beslutninger vil blive vægtet tungere end sonografernes argumenter.  
 
\section{Delkonklusion }
Set fra de patientmæssige forhold vil indførslen af en Ultralyds Robotarm ikke have en stor indflydelse på de gravide. Den usikkerhed og eventuel ubehag, der kan fremkomme, kan afhjælpes af sonografernes autoritet og deres tilgang til opgaven. Så længe den gravide har mulighed for at danne et forhold til fosteret, burde de gravide ikke have et problem med det nye udstyr. 
Set fra sonografernes synsvinkel kan indførslen af Robotarmen forbedre deres arbejdsforhold. 