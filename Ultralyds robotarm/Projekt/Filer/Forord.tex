\chapter{Forord}
Denne mini-MTV er udarbejdet på 4 semester på Sundhedsteknologi, Aarhus Universitet Ingeniørhøjskolen (IHA), under vejledning af Lene Häuser og Samuel Thrysøe. Projektet tager sit udgangspunkt i undervisningen på IHA, og udspringer fra kurset Medicinsk Teknologi Vurdering. 


I forbindelse med udformningen af denne Mini Medicinsk Teknologi Vurdering vil vi gerne takke følgende personer og afdelinger, for deres venlighed og hjælp.


Tak til Tina Arnbjørn fra Hospitalsenheden Horsens, Kvindeafdelingen, Svangre- og Ultralydsambulatorium for at stille op til interview og demonstrering af ultralyds scanninger.


Tak til Karen Marie Goul fra Regionshospital Midt Viborg, Kvindesygdomme og Fødsler for at stille op til interview. Samt et tak til afdelingen for demonstrationer af ultralydsscanninger. 


Endvidere skal der lyde en tak til Søren Holm Pallesen for at stille hans viden til rådighed omkring Ultralyds Robotarmen, samt inspiration til litteratursøgning. 

