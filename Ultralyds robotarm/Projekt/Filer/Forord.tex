\chapter{Forord}
Denne mini-MTV er udarbejdet på 4. semester på Sundhedsteknologi, Aarhus Universitet Ingeniørhøjskolen (IHA), under vejledning af Lene Häuser Petersen og Samuel Alberg Thrysøe. Projektet tager sit udgangspunkt i undervisningen på IHA, og udspringer fra kurset Medicinsk Teknologi Vurdering. 


I forbindelse med udformningen af denne mini Medicinsk Teknologi Vurdering vil vi gerne takke følgende personer og afdelinger, for deres venlighed og hjælp:


Tak til Tina Arnbjørn fra Hospitalsenheden Horsens, Kvindeafdelingen, Svangre- og Ultralydsambulatorium for at stille op til interview, og tak til afdelingen for demonstrering af ultralydsscanninger. 


Tak til Karen Marie Goul fra Regionshospital Midt Viborg, Kvindesygdomme og Fødsler for at stille op til interview. Samt et tak til afdelingen for demonstrationer af ultralydsscanninger. 


Endvidere skal der lyde en stor tak til Søren Pallesen for idegrundlag, hans viden omkring Ultralyds Robotarmen, samt inspiration til litteratursøgning. 

